\documentclass{article}
\usepackage[T1]{fontenc}
\usepackage{polski}
\usepackage[polish]{babel}
\usepackage[utf8x]{inputenc}
\usepackage{fontspec}
\usepackage{mathtools}
\usepackage{amssymb}
\usepackage[hidelinks]{hyperref}
\usepackage{amsmath,suetterl,graphicx,mathrsfs}
\usepackage[a4paper, total={6.2in, 10in}]{geometry}
\usepackage[skip=4pt plus1pt, indent=0pt]{parskip}
\usepackage{nicematrix}
\usepackage{physics}

\newcommand{\sign}{\text{sign}}

\title{Metody numeryczne zadanie 2}
\author{Bartosz Kucypera}
\date{\today}

\begin{document}

\maketitle

Niech $A \in \mathbb{R}^{N \times N}$ będzie nieosobliwą macierzą trójdiagonalną.

\section*{Rozkład QR macierzy A przekształceniami Householdera}
Wykorzystam zywkły algorytm znajdujący rozkład $QR$ macierzy (o którego poprawności wiemy już z ćwiczeń) i wykorzystam specyficzną strukturę $A$ by działał on w $O(N^2)$.

\subsection*{Wyzerowanie pierwszej kolumny pod diagonalą}
Niech $e =  \begin{bmatrix} 1 \\ 0 \\ \vdots \\ 0 \end{bmatrix}$, $\norm{\cdot}$ normą euklidesową i $x$ to zerowana kolumna.

Wyliczamy przekształcenie Householdera: 

$ \alpha = -\norm{x}* \sign(x_1)$

$ u = x - \alpha e$

$ v = \frac{u}{\norm{u}} $

$Q_1 = I - 2vv^T$

Wektor $x$ miał co najwyżej dwie niezerowe współżędne (pierwszą i drugą), czyli macierz $2vv^T$ ma co najwyżej niezerowy kwadrat $2\times2$ w lewym górym rogu. \newline
W takim razie domnożenie $Q_1$ do innej macierzy możemy robić liniowym kosztem. \newline
Po domnożeniu, zmieniają się co najwyżej dwa wiersze (lub dwie kolumny w zależności z której strony domnażamy).

Macierzy $Q_1$ nie potrzebujemy do niczego innego niż do domnażania jej (raz do macierzy na której pracujemy by uzyskać $R$ i raz na boku by uzyskać całe złożenie przekształceń Householdera, $Q$), możemy więc trzymać tylko cztery elementy macierczy $-2vv^T$ które mogą być niezerowe i przy domnażaniu odpowiednio modyfikować macierz.

\newpage

\subsection*{Algorytm}

Zerujemy pierwszą kolumnę przekształceniem $Q_1$. \newline
$P = Q_1 \cdot A$ \newline

$$ P = 
\begin{pNiceArray}{c|cccc}
    a_{11} & * & \cdots & * &\\
    \hline
    \Block{2-1}<\Large>{0} & \Block{2-4}<\Large>{P'} & & &\\
                           & & & & \\
\end{pNiceArray}
$$

$P'$ dalej jest trójdiagonalna (zmienić mógł jej się tylko pierwszy element na diagonali), więc możemy znaleźć rekurencyjnie jej rozkład $QR$.

Niech $P' = Q'R'$. \newline

Wtedy macierz 

$$\begin{pNiceArray}{c|cccc}
    a_{11} & * & \cdots & * &\\
    \hline
    \Block{2-1}<\Large>{0} & \Block{2-4}<\Large>{R'} & & &\\
                           & & & & \\
\end{pNiceArray} 
$$

jest szukaną macierzą $R$, a macierz

$$Q_1 \cdot \begin{pNiceArray}{c|cccc}
    1 & \Block{1-4}<\Large>{0} & & &\\
    \hline
    \Block{2-1}<\Large>{0} & \Block{2-4}<\Large>{Q'} & & &\\
                           & & & & \\
\end{pNiceArray}^T
$$
jest szukaną macierzą $Q$.

Macierz $R$ możemy wyliczać w miejscu, a macierz $Q$ możemy na początku ustawić jako identyczność i w trakcie wykonywania algorytu na bierząco domnażać do niej kolejne przekształcenia Householdera.

Wykonujemy wtedy $N$ kroków i w każdym z nich dwa razy domnażamy macierz przekształcenia Householdera do innej macierzy w czasie $O(N)$ (dzięki jej specyficznej strukturze), co łącznie daje nam czas działania $O(N^2)$.

\newpage

Kod w matlabie:

\begin{verbatim}
% QR Factorization by Householder reflections for strongly Tri Diagonal matrix
function [Q, R] = QRFHTD(A)
    N = size(A, 1);
    Q = eye(N);
    R = A;

    % zapominamy już o rekurencji, żeby łatwiej się implementowało
    % wszystkie przekształcenia wyliczamy odrazu w pełnym wymiarze i na bierząco domnażamy do Q
    for i = 1:(N-1)
        % wyliczenie wektora v
        alf = -sqrt(R(i, i).^2 + R(i+1, i).^2)*sign(R(i,i));
        uii = R(i, i) - alf;
        uji = R(i+1, i);
        nrm_u = sqrt(uii.^2 + uji.^2);
        uii = uii/nrm_u; % pierwsza współżędna wektora v
        uji = uji/nrm_u; % druga współżędna wektora v

        % zmienia nam się tylko sześć komórek w tym jedna z macierzy P' (P' z opisu algorytmu)
        for j = i:min(N, i+2)
            % zmiana komórek wierszy
            ch1 = -2 * (R(i, j)*uii*uii + R(i+1, j)*uii*uji);
            ch2 = -2 * (R(i, j)*uji*uii + R(i+1, j)*uji*uji);
            R(i, j) += ch1;
            R(i+1, j) += ch2;
        end

        % to samo tylko tym razem domnażamy przekształcenie Householdera z drugiej strony macierzy
        % czyli zmieniają nam się kolumny
        % tym razem może zmieniać się więcej komórek ale nie więcej niż O(N)
        for k = 1:N
            ch1 = -2 * (Q(k, i)*uii*uii + Q(k, i+1)*uji*uii);
            ch2 = -2 * (Q(k, i)*uii*uji + Q(k, i+1)*uji*uji);
            Q(k, i) += ch1;
            Q(k, i+1) += ch2;
        end
    end
end
\end{verbatim}


\end{document}
