\documentclass{article}
\usepackage[T1]{fontenc}
\usepackage{polski}
\usepackage[polish]{babel}
\usepackage[utf8x]{inputenc}
\usepackage{fontspec}
\usepackage{mathtools}
\usepackage{amssymb}
\usepackage[hidelinks]{hyperref}
\usepackage{amsmath,suetterl,graphicx,mathrsfs}
\usepackage[a4paper, total={6.2in, 10in}]{geometry}
\usepackage[skip=4pt plus1pt, indent=0pt]{parskip}
\usepackage{nicematrix}
\usepackage{physics}

\newcommand{\sign}{\text{sign}}

\title{Metody numeryczne zadanie 2}
\author{Bartosz Kucypera}
\date{\today}

\begin{document}

\maketitle

Niech $A \in \mathbb{R}^{N \times N}$ będzie nieosobliwą macierzą trójdiagonalną.

\section*{Rozkład QR macierzy A przekształceniami Householdera}
Wykorzystam zywkły algorytm znajdujący rozkład $QR$ macierzy (o którego poprawności wiemy już z ćwiczeń) i wykorzystam specyficzną strukturę $A$ by działał on w $O(N^2)$.

\subsection*{Wyzerowanie pierwszej kolumny pod diagonalą}
Niech $e =  \begin{bmatrix} 1 \\ 0 \\ \vdots \\ 0 \end{bmatrix}$, $\norm{\cdot}$ normą euklidesową i $x$ to zerowana kolumna.

Wyliczamy przekształcenie Householdera: 

$ \alpha = -\norm{x}* \sign(x_1)$

$ u = x - \alpha e$

$ v = \frac{u}{\norm{u}} $

$Q_1 = I - 2vv^T$

Wektor $x$ miał co najwyżej dwie niezerowe współżędne (pierwszą i drugą), czyli macierz $2vv^T$ ma co najwyżej niezerowy kwadrat $2\times2$ w lewym górym rogu. \newline
W takim razie domnożenie $Q_1$ do innej macierzy możemy robić liniowym kosztem. \newline
Po domnożeniu, zmieniają się co najwyżej dwa wiersze (lub dwie kolumny w zależności z której strony domnażamy).

Macierzy $Q_1$ nie potrzebujemy do niczego innego niż do domnażania jej (raz do macierzy na której pracujemy by uzyskać $R$ i raz na boku by uzyskać całe złożenie przekształceń Householdera, $Q$), możemy więc trzymać tylko cztery elementy macierczy $-2vv^T$ które mogą być niezerowe i przy domanżania odpowiednio modyfikować macierz.

\newpage

\subsection*{Algorytm}

Zerujemy pierwszą kolumnę przekształceniem $Q_1$. \newline
$P = Q_1 * A$ \newline

$$ P = 
\begin{pNiceArray}{c|cccc}
    a_{11} & * & \cdots & * &\\
    \hline
    \Block{2-1}<\Large>{0} & \Block{2-4}<\Large>{P'} & & &\\
                           & & & & \\
\end{pNiceArray}
$$

$P'$ dalej jest trójdiagonalna (zmienić mógł jej się tylko pierwszy element na diagonali), więc możemy znaleźć rekurencyjnie jej rozkład $QR$.

Niech $P' = Q'R'$. \newline

Wtedy macierz 

$$\begin{pNiceArray}{c|cccc}
    a_{11} & * & \cdots & * &\\
    \hline
    \Block{2-1}<\Large>{0} & \Block{2-4}<\Large>{R'} & & &\\
                           & & & & \\
\end{pNiceArray} 
$$

jest szukaną macierzą $R$, a macierz

$$\begin{pNiceArray}{c|cccc}
    1 & \Block{1-4}<\Large>{0} & & &\\
    \hline
    \Block{2-1}<\Large>{0} & \Block{2-4}<\Large>{Q'} & & &\\
                           & & & & \\
\end{pNiceArray}^T * Q_1
$$
jest szukaną macierzą $Q$.

Macierz $R$ możemy wyliczać w miejscu, a macierz $Q$ możemy na początku ustawić jako identyczność i w trakcie wykonywania algorytu na bierząco domnażać do niej kolejne przekształcenia Householdera.

Wykonujemy wtedy $N$ kroków i w każdym z nich dwa razy domnażamy macierz przekształcenia Householdera do innej macierzy w czasie $O(N)$ (dzięki jej specyficznej strukturze), co łącznie daje nam czas działania $O(N^2)$.

{\bf Implementacja algorytmu w pliku QRHTD.m}
\end{document}
