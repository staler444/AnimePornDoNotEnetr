\documentclass{article}
\usepackage[T1]{fontenc}
\usepackage{polski}
\usepackage[polish]{babel}
\usepackage[utf8x]{inputenc}
\usepackage{fontspec}
\usepackage{mathtools}
\usepackage{amssymb}
\usepackage[hidelinks]{hyperref}
\usepackage{amsmath,suetterl,graphicx,mathrsfs}
\usepackage[a4paper, total={7in, 10in}]{geometry}
\usepackage[skip=4pt plus1pt, indent=0pt]{parskip}

\title{Druga praca domowa}
\author{Bartosz Kucypera}
\date{\today}

\begin{document}

\maketitle

\section*{Zadanie 2}
\subsection*{a)}
Wykaż, że jeśli współczynniki $b_0, b_1, b_2$ rozwinięcia w bazie Newtona wielomianu interpolacyjnego Lagrange'a opartego na trzech węzłach równoodległych: $x_0=0, x_1 =1,x_2 =2$ zaburzymy z błędem bezwzględnym nie przekraczającym $\epsilon$, to jego wartości na przedziale $[x_0, x_2]$ zmienią się nie więcej niż o $E=5\epsilon$.

Wielomian interpolacyjny Lagrange'a oparty na węzłach $x_0, x_1, x_2$ to
$$b_0 + b_1(x-x_0) + b_2(x-x_0)(x-x_1)$$
jeśli zaburzymy współczynniki $b_0,b_1,b_2$ o $\epsilon_0, \epsilon_1, \epsilon_2$ ($|\epsilon_0|, |\epsilon_1|, |\epsilon_2| \le \epsilon $) to błąd bezwzględny, dla danego $x$ wyniesie:

$$\left|(b_0+\epsilon_0) + (b_1+\epsilon_1)(x-x_0) + (b_2+\epsilon)(x-x_0)(x-x_1) - \left(b_0 + b_1(x-x_0) + b_2(x-x_0)(x-x_1)\right) \right| $$
czyli:
$$\left|\epsilon_1 + \epsilon_1(x-x_0) + \epsilon_2(x-x_0)(x-x_1)\right|$$
możemy skorzystać z nierówności trójkąta:
$$\left|\epsilon_1 + \epsilon_1(x-x_0) + \epsilon_2(x-x_0)(x-x_1)\right| \le |\epsilon_1| + |\epsilon_1(x-x_0)| + |\epsilon_2(x-x_0)(x-x_1)|  $$
i każdy składnik sumy oszacować osobno.
\subsubsection*{1) $|\epsilon_0|$}
Z treści mamy 
$$|\epsilon_0| \le \epsilon$$
\subsubsection*{2) $|\epsilon_1(x-x_0)|$}
Jest to moduł z funkcji liniowej, więc na przedziale $[x_0,x_2]$ osiąga maksimum w jednym z krańców przedziału. W $x_0$ się zeruje, więc maksimum w $x_2$.
$$|\epsilon_1(x_2-x_0)| = |\epsilon_1\cdot2| \le 2\epsilon$$

\subsubsection*{3) $|\epsilon_2(x-x_0)(x-x_1)|$}
Jest to moduł z funkcji kwadratowej, czyli na przedziale $[x_0, x_2]$ osiąga maksimum w jednym z krańców przedziału, lub w wierzchołku paraboli pod modułem. W $x_0$ się zeruje, więcj od razu ten punkt odrzucamy.

W $x_w$:
$$x_w = \frac{x_0 + x_1}{2}$$
$$|\epsilon_2(x_w-x_0)(x_w-x_1)| = \left|\epsilon_2\frac{1}{2}\frac{-1}{2}\right| \le \frac{1}{4}\epsilon$$
W $x_2$:
$$|\epsilon_2(x_2-x_0)(x_2-x_1)| = |\epsilon_2\cdot2\cdot1| \le 2\epsilon$$
czyli maksimum jest z góry ograniczone przez $2\epsilon$.


Czyli całą sumę możemy ograniczyć z góry:
$$|\epsilon_1| + |\epsilon_1(x-x_0)| + |\epsilon_2(x-x_0)(x-x_1)| \le \epsilon + 2\epsilon + 2\epsilon = 5\epsilon  $$

\newpage

\subsection*{b)}
Oszacuj $E$ dla przypadku, gdy $x_i=i⋅h$ $(i=0,1,2)$ dla pewnego $h>0$.

Zauważmy, że w podpunkcie $a)$, kożystaliśmy z wartości punktów $x_0, x_1, x_2$ dopiero przy obliczaniu maksimum składowych sum na przedziale $[x_0, x_2]$.

Zadanie sprowadza się, więc do ponowenego obliczenia maksimów z innymi wartościami punktów.

\subsubsection*{1)$|\epsilon_0|$}
Nic się nie zmienia.
$$|\epsilon_0| \le \epsilon$$

\subsubsection*{2)$|\epsilon_1(x-x_0)|$}
Znowu w $x_0$ wartość $0$, więc maksimum w $x_2$.
$$|\epsilon_1(x_2-x_0)| = |2h\epsilon_1| \le 2h\epsilon$$

\subsubsection*{3)$|\epsilon_2(x-x_0)(x-x_1)|$}
W $x_0$ wartość 0, maksimum w $x_2$ lub $x_w$.

W $x_w$:
$$ x_w = \frac{x_0+x_1}{2} = \frac{h}{2}$$
$$|\epsilon_2(x_w-x_0)(x_w-x_1)| = \left| \epsilon_2\frac{h}{2}\frac{-h}{2} \right| \le \frac{h^2}{4}\epsilon$$
W $x_2$:
$$|\epsilon_2(x_2-x_0)(x_2-x_1) = |\epsilon_2\cdot2h\cdot h| \le 2h^2\epsilon$$
czyli maksimum ograniczone z góry przez $2h^2\epsilon$.

Podsumowując $$E \le 2h^2\epsilon + 2h\epsilon + \epsilon.$$

\end{document}
