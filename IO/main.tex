\documentclass{article}
\vskip -1ex

\usepackage[parfill]{parskip}
\usepackage{multicol}
\usepackage{graphicx}

\usepackage{hyperref}
\hypersetup{
    colorlinks=true,
    linkcolor=blue,
    filecolor=magenta,      
    urlcolor=cyan,
    pdftitle={Overleaf Example},
    pdfpagemode=FullScreen,
    }

\urlstyle{same}

\title{\textbf{Architektura aplikacji}}

\begin{document}
\maketitle

\section{Serwer}

\subsection{Technologie dotyczące części aplikacyjnej}

Serwer aplikacji zostanie napisany w języku Rust z użyciem technologii Axum. Będzie udostępniał endpointy, do których będą mogły być wysyłane zapytania REST API. Serwer będzie pośrednikiem pomiędzy użytkownikiem a bazą danych i będzie jedynym miejscem, z którego będą mogły być wykonywane zapytania bezpośrednio na bazie danych.

\subsection{Technologie dotyczące komunikacji z bazą danych}

Komunikacja zostanie zrealizowana z użyciem biblioteki SQLx. Zapytania będą konstruowane w języku SQL, a potem importowane jako argumenty odpowiednich funkcji lub makr.

\subsection{Udostępnione endpointy}
Dokumentacja api udostępnianego przez serwer znajduje się pod tym \href{https://fuchczyk.github.io/io-swagger/}{linkiem}.
\iffalse
%%
Ilekroć w dokumentacji endpointów jest mowa o obiekcie reprezentującym lek (lub leku) to mamy na myśli następującą strukturę:
\begin{minted}[frame=single,
               framesep=3mm,
               xleftmargin=21pt,
               tabsize=4]{js}
{
    "id" : Number, // Integer
    "name": String,
    "substance" : String,
    "content": String,
    "ean": String,
    "selling_price": Number, // Number with two decimal places.
    "wholesale_price": Number, // Number with two decimal places.
    "retail_price": Number // Number with two decimal places.
}
\end{minted}

\textbf{Endpoint}: \verb|/search| \\
\textbf{Opis}: Wyszukiwanie leku w bazie danych. \\
W polu \textit{phrase} należy przesłać wpisaną frazę wyszukiwania.
\begin{multicols}{2}
\begin{minted}[escapeinside=||]{js}
{ // Request data.
    "Phrase": String
}
\end{minted}\breakd
\begin{minted}[escapeinside=||]{js}
{ // Response data.
    "Result": Array<Lek>
}
\end{minted}
\end{multicols}

\textbf{Endpoint}: \verb|/equivalents/life| \\
\textbf{Opis}: Wyznaczanie odpowiedników w terapii długotrwałej.

\textbf{Endpoint}: \verb|/equivalents/single| \\
\textbf{Opis}: Wyznaczanie odpowiedników w terapii krótkotrwałej.

\textbf{Endpoint}: \verb|/drug/payment| \\
\textbf{Opis}: Sprawdzanie możliwych poziomów odpłatności.

%\textbf{Endpoint}: \verb|/drug/diseases| \\
%\textbf{Opis}: Sprawdzenie możliwych zakresów wskazań chorobowych.
%%
\fi
\section{Interfejs użytkownika}

Interfejs użytkownika zostanie stworzony z użyciem HTML5, CSS3 oraz JavaScript. Strony opisane w specyfikacji (to znaczy Strona Główna, Wybór Choroby, Lista wyników) to będą strony z odpowiednimi wartościami metody GET. Ściślej, będą zgodne z następującym GUI:
\begin{itemize}
    \item 'GET /' - kod HTML strony głównej.
    \item 'GET /search?q=<fraza>$\&$p=<strona>'- kod HTML na zapytanie przekazane w parametrze 'q' i numerze strony przekazanym w parametrze 'p'.
    \item 'GET /selection/<ean>' - strona z wyborem poziomu refundacji odpowiadająca produktowi o kodzie EAN '<ean>'.
    \item 'GET /<id>' - strona z dokładnymi informacjami o leku i jego odpowiednikach (po wyborze poziomu refundacji) odpowiadająca produktowi, któremu zostało przypisane `<id>` wewnątrz
\end{itemize}

% Część aplikacji odpowiedzialna za interaktywność zostanie utworzona przy pomocy frameworka React.

\subsection{Strona główna}
% \begin{center}
%     \includegraphics[width=0.7\textwidth]{site/strona_glowna.png}
% \end{center}
Użytkownik będzie wpisywał do etykiety <input type='search'> (znajdującego się na środku strony) wyrażenie do wyszukiwania.
Po kliknięciu przycisku ENTER będzie wysyłane zapytanie do serwera na adres '/api/search'. Wynikiem będzie posortowana lista leków na podstawie algorytmu opisanego w 
\href{https://github.com/KamilloP/IO/blob/main/algos_io.png}{specyfikacji}.
\iffalse
%%
\begin{center}
    \includegraphics[width=0.7\textwidth]{site/algos_io.png}
\end{center}
\fi
\subsection{Wyniki wyszukiwania}
% \begin{center}
%     \includegraphics[width=0.7\textwidth]{site/po_wyszukaniu.png}
% \end{center}
Wyświetlona zostanie tabela wyników, maksymalnie 50 wyników na raz, w przypadku,
gdy miałoby być zwróconych ich więcej, na dole strony będzie się znajdowała Nawigacja
stron z wynikami. Po najechaniu kursorem na dany wiersz tabeli wyników, zostanie on podświetlony, sugerując że może zostać kliknięty przez użytkownika (zostanie użyta opcja HTML 'hover' oraz CSS w tym celu). Po kliknięciu odpowiedniej etykiety <input type='button'>, w przypadku posiadania przez lek wielu różnych wskazań refundacyjnych, użytkownik zostanie przeniesiony do okna wyboru choroby. Kliknięcie wyśle do serwera zapytanie na adres '/api/groups/<group$\_$id>.json'.
\subsubsection{Nawigacja stron}
Po kliknięciu kursorem etykiety <input type='button'>, użytkownik zostanie przeniesiony na odpowiedni fragment listy wyników (część wyników będzie niewidoczna dla użytkownika, bo w HTML będzie oznaczona jako 'hidden', JS zmieni widoczność).
\subsection{Wybór choroby}
Po najechaniu kursorem na dany wiersz tabeli wyników, zostanie on podświetlony.
 Po kliknięciu (odpowieda etykiecie <input type='button'>) przeglądarka wyśle zapytanie do serwera na adres '/api/selection/<ean>.json'.
\subsection{Dokładne informacje o leku}
Po najechaniu kursorem na dany wiersz tabeli wyników, zostanie on podświetlony.
 Po kliknięciu (odpowieda etykiecie <input type='button'>) przeglądarka wyśle zapytanie do serwera na adres '/api/details/<id>.json'.
 
\section{Baza danych}

Dane przetwarzane przez projekt będą obsługiwane przez bazę danych PostgreSQL. Tabele w bazie będą przechowywane w schemacie utworzonym przez polecenie:
\begin{verbatim}
    CREATE SCHEMA odpowiedniki;
\end{verbatim}
Tabela zawierająca skatalogowane leki:
    \begin{verbatim}
    CREATE TABLE odpowiedniki.leki (
        id INTEGER PRIMARY KEY,
        name VARCHAR NOT NULL,
        ean VARCHAR NOT NULL,
        refund VARCHAR NOT NULL,
        price INTEGER NOT NULL,
        amount INTEGER NOT NULL,
        amount_unit VARCHAR,
        group_id INTEGER NOT NULL
    );
    \end{verbatim}
W przypadku leku, pole \verb|"amount_unit"| będzie zawierało jednostkę w której została podana ilość sztuk w opakowaniu. 
Jeśli lek nie został przetworzony, pole \verb|"amount_unit"| pozostaje puste, oraz pole \verb|"group_id"| zawiera wartość -1.

\subsection{Zapytania bazy danych}

Interfejs użytkownika będzie realizowany za pomocą następujących zapytań w bazie danych: \newline

\centerline{GET /search?q=<fraza>&p=<strona>}




\end{document}
