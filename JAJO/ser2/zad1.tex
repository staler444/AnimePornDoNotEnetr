\documentclass{article}
\usepackage[T1]{fontenc}
\usepackage{polski}
\usepackage[polish]{babel}
\usepackage[utf8x]{inputenc}
\usepackage{fontspec}
\usepackage{mathtools}
\usepackage{amssymb}
\usepackage[hidelinks]{hyperref}
\usepackage{amsmath,suetterl,graphicx,mathrsfs}
\usepackage[a4paper, total={7in, 11in}]{geometry}
\usepackage[skip=4pt plus1pt, indent=0pt]{parskip}

\title{JAO praca domowa}
\author{Bartosz Kucypera, bk439964}
\date{\today}

\begin{document}

\maketitle

\section*{Zadanie 2.1}
$$ L_{\exists} = \displaystyle \left\{ ab^{n_1}ab^{n_2} \dots ab^{n_k}a \in \{a, b\}^* \; 
| \; \exists i \in \mathbb{N}. \; 1 \le i \le k \land  n_i = k \right\}$$

\subsection*{Definicja gramtyki}

Niech $ \mathcal{G} $ będzie gramatyką bezkontekstową opisującą $ L_{\exists}$. $ \mathcal{G} = (T, N, P, S) $ gdzie:
\begin{itemize}
	\item[] $T$ - zbiór symboli terminalnych, $T = \{a, b\}$
	\item[] $N$ - zbiór symboli nieterminalnych, $N = \{S, L_0, R_0, L_1, R_1 \}$
	\item[] $P$ - zbiór reguł, zdefiniowany poniżej
	\item[] $S$ - symbol początkowy
\end{itemize}
Do $P$ należą następujące reguły:
$$ S \to L_0bR_0 \;| \; a$$
$$ L_0 \to aL_1b \; | \; a$$
$$ R_0 \to bR_1a \; | \; a$$
$$ L_1 \to aL_1b \; | \; bL_1 \; | \; a$$
$$ R_1 \to bR_1a \; | \; R_1b \; | \; a$$

\subsection*{Inkluzja $L_{\exists} \subseteq L(\mathcal{G})$}

Niech $w$ będzie dowolnym słowem z $L_{\exists}$. Pokażemy, że potrafimy zwinąć $w$ do $S$ korzystając z odwrotnych przejść gramatyki $\mathcal{G}$, czyli, że $w \in L(\mathcal{G})$. \newline
Niech $k$ będzie stałą z definicji $w$, oraz niech $a_l, a_r$ będą zwykłymi literami $a$ z $w$, wyróżnionymi dla naszej wygody.
$$w = prefix \cdot a_lb^ka_r \cdot suffix$$ (pomijając trywialny przypadek dla $w = a$). \newline
Pokażmy, jak zwinąć $prefix$, $suffix$ zwija się analogicznie. \newline
Jeśli na początku $|prefix|=0$, to przekształcamy $a_l \to L_0$ i kończymy. \newline \newline
Jeśli nie to przekształcamy $a_l \to L_1$ i wykonujemy następujący algorytm: \newline 
\begin{itemize}
	\item[] Jeśli $prefix = a$, to przekształcamy $aL_1b \to L_0$ i kończymy.
	\item[] Jeśli $prefix = prefix' \cdot a$ to przekształcamy $aL_1b \to L_1$, czyli $prefix$ traci ostatnią literkę i "zjadamy" jedno z $b$ pomiędzy $a_l$ i $a_r$.
	\item[] Jeśli $prefix = prefix' \cdot b$ to przekształcamy $bL_1 \to L_1$, czyli $prefix$ po prostu traci ostatnią literkę.
\end{itemize}

Algorytm zawsze się skończy, bo $prefix$ jest skończony i zaczyna się od literki $a$. \newline
Analogicznie postępujemy dla $suffixu$. \newline
Pozbyliśmy się już $suffixu$ i $prefixu$. Teraz zauważmy, że początkowo w $w$ mieliśmy $k+1$ liter $a$ (z definicji $w$), dwie wyróżniliśmy ($a_l$ i $a_r$) a pozostałe $k-1$ zjadł nasz algorytm. Skoro każde usunięcie $a$ wiązało się też ze "zjedzeniem" $\;$ jednej z liter $b$ pomiędzy $a_l$ i $a_r$ których na początku było $k$, to została nam tylko jedna. \newline 
Zwineliśmy, więc $w$ do $L_0bR_0$. Wystarczy wykonać teraz ostatni krok i przekształcić $L_0bR_0$ do $S$.\newline \newline Skoro umiemy ciągiem operacji odwrotnych zwinać $w$ do $S$ to niewątpliwie potrafimy je też wygenerować za pomocą gramatyki $\mathcal{G}$. Z dowolności wyboru $w$ wnioskujemy, że $L_{\exists} \subseteq L(\mathcal{G})$.

\newpage

\subsection*{Inkluzja $ L(\mathcal{G}) \subseteq L_{\exists}$}
Pokażmy indukcyjnie, że jesteśmy w stanie generować jedynie słowa należące do $L_{\exists}$. \newline \newline
Przypadek bazowy: $L_0bR_0$ (jeśli $S \to a$ to oczywiście $a \in L_{\exists}$ ok), \newline 
kończymy dalsze generowanie, zamieniamy $L_0 \to a, R_0 \to a$, $aba \in L_{\exists}$ ok. \newline 

Krok indukcyjny: \newline
Załóżmy, że wygenerowaliśmy poprawne wyrażenie postaci: 
$$prefix \cdot L_xb^kR_y \cdot suffix $$
*$L_x$ to $L_0$ lub $L_1$, $R_y$ to $R_0$ lub $R_1$. \newline
*poprawne - do $prefix \cdot suffix$ należy dokładnie $k-1$ liter $a$ i po zamianie $L_x$ i $R_y$ na $a$, słowo będzie należało do $L_{\exists}$. \newline\newline
Możemy wykorzystać jedno z dwóch przekształceń (oprócz tych kończących, zamieniających w $a$):
$$L_x \to aL_1b \mbox{ albo } R_y \to bR_1a.$$
Pokażmy dla $L_x$, dla $R_y$ dowód jest analogiczny.
Przekształcamy: 
$$prefix \cdot L_xb^kR_y \cdot suffix \to prefix \cdot aL_1b^{k+1}R_y \cdot suffix $$

Nie dodaliśmy do początku żadnej literki $b$, wcześniejsze słowo maiło $k+1$ liter $a$ (razem z $L_x$ i $R_y$), nowe ma więc $k+2$, ale ma też blok liter $b$ długości $k+1$, czyli wszystko się zgadza. Nowe wyrażenie też jest poprawne* i też jest postaci: $$prefix'\cdot L_xb^{k'}R_x \cdot suffix$$
dla $prefix' = prefix \cdot a$ i $k' = k+1$. \newline
Na mocy indukcji matematycznej, gramtyką $\mathcal{G}$ jesteśmy w stanie wygenerować tylko słowa należące do $L_{\exists}$, czyli zachodzi $ L(\mathcal{G}) \subseteq L_{\exists}$.

\subsection*{Konkluzja}
Skoro
$$L_{\exists} \subseteq L(\mathcal{G}) \land L(\mathcal{G}) \subseteq L_{\exists}$$
to zachodzi
$$ L_{\exists} = L(\mathcal{G}) $$
czyli gramatyka $\mathcal{G}$ faktycznie opisuje $L_{\exists}$. Gramatyka $\mathcal{G}$ jest bezkontekstowa, więc oczywiście $L_{\exists}$ jest językiem bezkontekstowym.

\end{document}
