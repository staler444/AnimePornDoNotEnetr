\documentclass{article}
\usepackage[T1]{fontenc}
\usepackage{polski}
\usepackage[polish]{babel}
\usepackage[utf8x]{inputenc}
\usepackage{fontspec}
\usepackage{mathtools}
\usepackage{amssymb}
\usepackage[hidelinks]{hyperref}
\usepackage{amsmath,suetterl,graphicx,mathrsfs}
\usepackage[a4paper, total={6.2in, 10in}]{geometry}
\usepackage[skip=4pt plus1pt, indent=0pt]{parskip}

\title{JAO seria 3}
\author{Bartosz Kucypera}
\date{\today}

\begin{document}

\maketitle

\section*{Zadanie 3}

/* wszystkie oznaczenia jak w treści */ \newline
/* jeśli $u_i$ kodem maszyny turinga z treści, to $p_i$ programem wykonywanym przez tą maszynę */ \newline
/* zakłdamy, że algorytm nie musi być skończony dla każdych danych, \newline
ale zakładamy, że umiemy go zaprogramować w maszynie turinga */

$a)$ Czy język $L$ jest obliczalny? \newline
$b)$ Czy język $L$ jest częściowo obliczalny? \newline
$c)$ Czy dopełnienie języka $L$ jest częściowo obliczalne? 

\subsection*{}
Zauważmy, że jeśli $b)$ i $c)$ są prawdziwe, to $a)$ też jest prawdziwe. \newline
Jeśli $L$ jest częściowo obliczalny to istnieje program, który dla danej pary $<u_1,u_2>$,
jeśli należy ona do $L$, wykaże to w skończonym czasie. Analogicznie dla dopełnienia $L$.\newline
Jeśli zastanawiamy się czy dana para $<u_1, u_2>$ należy do $L$, możemy uruchomić te programy współbieżnie, np. wykonując po jednym kroku w każdym, na zmiane. Jeśli jeden z nich się zakończy to będziemy wiedzieć czy $<u_1,u_2>$ należy do $L$. Zawsze któryś z nich się zakończy, bo każda para $<u_1, u_2>$, należy albo do $L$, albo do jego dopełnienia. \newline
\newline 
Skoro z prawdziwości $[b) \land c)]$ wynika prawdziwość $a)$, to jeśli wykażemy, że $a)$ nie jest prawdziwe, oraz że $c)$ jest prawdziwe, to wiemy też, że $b)$ nie jest prawdziwe. Jeśli $b)$ było by prawdziwe, to z wcześniejszej implikacji, $a)$ też by było, czyli sprzeczność.

\subsection*{Nieprawdziwość $a)$}
Załóżmy, że $a)$ prawdziwe. Zauważmy, że w takim razie jesteśmy wstanie rozwiązać nierozwiązywalny problem $stopu$.
Pokażmy, że dla każdego programu i każdego wejścia jesteśmy wstanie stwierdzić czy program się zatrzyma. \newline
Niech $p_1$ rozważanym programem i niech $s$ rozważanym słowem wejściowym. \newline
Niech teraz $p_2$, i $p_3$ takimi programami, że $p_2(s) = 0$ i $p_3(s) = 1$. Dla reszty danych wejściowych $p_2$ i $p_3$ nie zatrzymują się.
Wystarczy teraz sprawdzić czy pary kodów $<u_1, u_2>$, $<u_1, u_3>$ należą do $L$. Robimy to w skończonym czasie, z założenia. Jeśli obie pary należą do $L$ to niewątpliwie $p_1$ nie zatrzymuje się dla $s$. Jeśli zatrzymywało by się i $p_1(s) = s'$ to mamy $p_2(s)=p_1(s)=p_3(s)$ czyli $0=s'=1$, sprzeczność.
Jeśli tylko jedna para należy, to $p_1$ zatrzymuje się i $s'=0$ lub $s'=1$, jeśli żadna para nie należy to $p_1$ zatrzymuje się i $s' \neq 0$ i $s' \neq 1$. \newline
Opisany algorytm jest zawsze skończony, gdyż, kody $u_2$ i $u_3$ są skończone i można je bardzo łatwo wygenerować (jeden $if$ i pętla $while(true)$, przetłumaczone na kod maszyny turinga), natomiast obliczalość sprawdzenia czy pary kodów należą do $L$, wynika z założonej prawdziwości a). \newline 
Czyli faktycznie $a)$ nie może być prawdziwe, bo problem $stopu$ jest nierozwiązywalny. \newline

\subsection*{Prawdziwość $c)$}
Skonstruujmy algorytm, który dla każdej pary $<u_1, u_2>$, jeśli należy ona do dopełnienia $L$, stwierdzi to w skończonym czasie. \newline
Niech $<u_1, u_2>$ parą należącą do dopełnienia $L$.
Istnieje, więc takie słowo $s$, że $p_1$ i $p_2$ zatrzymują się na $s$ oraz, $p_1(s) \neq p_2(s)$.
Słów w $\{0,1\}^*$ jest przeliczalnie wiele, więc możemy je ponumerować, np. sortujac leksykograficznie. \newline
Nasz algorytm będzie działał tak, że będziemy na raz mieli "uruchomionych" $\:$ dużo kopi programów $p_1$ i $p_2$ z kolejnymi słowami z $\{0,1\}^*$ jako argumentami wejściowymi. 
Taśma maszyny turinga jest nieskończona, więc możemy mieć zapisane stany skończenie wielu kopi $p_1$ i $p_2$.
W każdej turze algorytmu, przechodzimy się po każdej kopi, ładujemy jej stan, wykonujemy kolejny krok programu, zapisujemy jej nowy stan. Dodatkowo na końcu tury tworzymy dwie świerze kopie $p_1$ i $p_2$, podając im jako argument wejściowy kolejne słowo z $\{0,1\}^*$.
Jeśli jakaś para kopi się zatrzyma (para w sęsie kopie $p_1$ i $p_2$ uruchomione na tym samym słowie), i da różne wyniki, to zatrzymujemy cały algorytm i zwracamy, że $<u_1, u_2>$ należy do dopełnienia $L$. Powtarzmy przebieg takiej tury dopóki nie napotkamy takiej pary. \newline \newline

Teraz jeśli wiemy, że istnieje takie słowo dla którego $p_1$ i $p_2$ dają różny wynik, to nasz algorytm na pewno się zatrzyma. Niech $s$ będzie takim słowem.
Zauważmy, że każda tura algorytmu wykonuje się w skończonym czasie. Dodajemy dwie kopie programów i dla skończonej ilości kopi wykonujemy po jednym kroku. Teraz skoro programy kończą się dla $s$ to istnieje skończona liczba kroków po wykonaniu których oba się skończą dla $s$ (max z liczby kroków dla $p_1$ i liczby kroków dla $p_2$). Niech $k$ będzie tą liczbą. \newline
Słowo $s$ ma jakiś skończony idenks w posortowaniu leksykograficznym. Niech $i$ będzie tym indeksem.
Wiemy teraz, że nasz algorytm zatrzyma się najpóźniej po $k$+$i$ turach. W $i$-tej turze dodamy kopie programów ze słowem wejściowym $s$. Po kolejnych $k$ turach obie kopie się zatrzymają, (będą zatrzymane, jedna mogła skończyć się wcześniej), i ponieważ $p_1(s) \neq p_2(s)$ cały nasz algorytm się zakończy, stwierdzając, że $<u_1, u_2>$ należy do dopełnienia $L$. \newline 

Algorytm można łatwo zaimplementować w jakimś normalnym języku programowania, czyli jest też to wykonalne na maszynie turinga, czyli dopełnienie $L$ faktycznie jest częściowo obliczalne.

\subsection*{Synteza}
Skoro z prawdziwości $\displaystyle \left[ b) \land c) \right]$ wynika prawdziwość $a)$, oraz zachodzi $[\lnot a) \land c)]$ to niewątpliwe zachodzi, też $[\lnot b)]$.

\end{document}
