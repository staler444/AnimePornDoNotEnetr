\documentclass{article}
\usepackage[T1]{fontenc}
\usepackage{polski}
\usepackage[polish]{babel}
\usepackage[utf8x]{inputenc}
\usepackage{fontspec}
\usepackage{mathtools}
\usepackage{amssymb}
\usepackage[hidelinks]{hyperref}
\usepackage{amsmath,suetterl,graphicx,mathrsfs}
\usepackage[a4paper, total={6.2in, 10in}]{geometry}
\usepackage[skip=4pt plus1pt, indent=0pt]{parskip}

\usepackage{amsfonts}
\usepackage{amsmath}
\usepackage{enumitem}
\usepackage{braket}
\usepackage{amssymb}
\usepackage{xcolor}
\usepackage{hyperref}
\usepackage{graphicx}
\usepackage{mathtools}
\usepackage{tikz}
\usetikzlibrary{graphs}
\usetikzlibrary{automata}
\usetikzlibrary{positioning}
\usetikzlibrary{quotes}

\title{JAO praca domowa}
\author{Bartosz Kucypera, bk439964}
\date{\today}

\begin{document}

\maketitle

\section*{Zadanie 1.1}

Dla danego alfabetu $A$ oraz języka $L \subseteq A^*$ zdefinujmy EvenLen($L$) jako 

$$\{w \in \{1\}^* |\mbox{ liczba słów długości $|w|$ w $L$ jest parzysta} \}$$

Wykaż, że klasa języków regularnych jest zamknięta ze względu na operację EvenLen. \newline
\newline 
Niech $L' = $ EvenLen($L$).
Skorzystajmy z faktu iż regularność języka, jest równoważna istnieniu NFA rozpoznającego dany język. Pokażemy, że korzystając z istnienia NFA dla $L$ jesteśmy w stanie skonstruować NFA rozpoznający $L'$. \newline 
\newline 
$L$ regularny, więc istnieje NFA rozpoznający go. Niech $A$ będzie takim automatem o stanach $q_1, q_2, q_3, ... q_n$. Bez staraty ogólności załóżmy, że stan $q_1$ jest stanem początkowym. Konstruujemy graf pomocniczy w następujący sposób: \newline
Będziemy konstruować go indukcyjnie, dokładając kolejne warstwy.
Zaczynamy od pierwszej warstwy zawierającej wierzchołki $q_{1,0}, q_{2,0}, ..., q_{n,0}$
Wierzchołek $q_{i,j} = $ wierzchołek odpowiadający $i$-temu stanowi z $A$ leżący na $j$-tej warstwie. \newline 
Krawędzie istnieją tylko pomiędzy sąsiednimi warstwami. Dokładając $k+1$-pierwszą warstwę, jeśli w $A$ istnieje krawędź z $q_i$ do $q_j$ to w grafie dodajemy krawędź z $q_{i,k}$ do $q_{j,k+1}$.

Każdemu wierzchołkowi przypisujemy wartościowanie, 0 lub 1. Jeśli wierzchołek $q_{i, j}$ ma wartościowanie 0, to znaczy, że możemy do niego dojść z wierzchołka $q_{1, 0}$ na parzyście wiele sposobów, w przeciwnym razie na nieparzyście. \newline
Dla pierwszej warstwy wierzchołek $q_{1,0}$ ma wartościowanie 1, cała reszta 0. Dla kolejnych warstw wartościowanie wierzchołka to suma wartościowań wierzchołków do niego wchodzących $\%2$ (zakładamy, że pusta suma daje 0). \newline
Dodatkowo niech wierzchołek $q_{i,j}$ będzie wierzchołkiem wyróżnionym, jeśli w $A$ stan $q_i$ był stanem akceptującym. \newline \newline

Zauważmy, że posiadając taki graf łatwo sprawdzić czy słowo o długości $l$ należy do $L'$. Weźmy sumę wartościowań wierzchołków wyróżnionych na warstwie $l$ i zmodulujmy ją przez $2$. Jeśli ta jest równa 1 to znaczy, że $A$ akceptował nieparzyście wiele słów długości $l$, czyli słowo długości $l$ nie należy do $L'$, (w przeciwnym wypadku akceptował parzyście wiele, czyli należy do $L'$). \newline \newline

Nasz graf ma kilka ważnych własności: \newline Połączenie pomiędzy kolejnymi warstwami są takie same, jeśli istnieje krawędź z $q_{i,k}$ do $q_{j,k+1}$ to istnieje bliźniacza z $q_{i,k+1}$ do $q_{j,k+2}$. \newline Jedyne krawędzie wychodzące z $k$-tej warstwy, biegną do $k+1$-rwszej warstwy. \newline 
Czyli wartościowanie $k+1$-rwszej warstwy zależy tylko i wyłącznie od wartościowania $k$-tej, co więcej jeśli warstwy $k_1$ i $k_2$ mają takie same wartościowania, to warstwy $k_1+1$ i $k_2+1$ też będą miały takie same wartościowania. \newline 
Wszystkich możliwych wartościowań jest co najwyżej $2^n$, bo $A$ miał $n$-stanów, czyli każda warstwa ma $n$ wierzchołków, a każdy wierzchołek ma 2 możliwe wartościowania. \newline 
Czyli po wyliczeniu co najwyżej $2^n$ warstw, wartościowania się zacyklą a tylko od nich zależy czy słowo danej długości jest akceptowane. \newline \newline 
Możemy więc już bez trudu skonstruować automat rozpoznający $L'$. \newline \newline 
Zaczynamy kładac wierzchołek początkowy odpowiadający pierwszej warstwie. Teraz wykonujemy następujący algorytm. Jeśli kolejna warstwa ma wartościowanie którego jeszcze nie napotkaliśmy, to dokładamy kolejny stan, dodajemy krawędź od poprzedniego stanu do nowego, i jeśli suma wyróżnionch wierzchołków na warstwie $\%2 = 0$ to nowy stan jest akceptujący. \newline 
Jeśli napotkaliśmy już dane wartościowanie, to nie dodajemy nowego stanu, tylko dokładamy krawędź od stanu aktualnego do stanu odpowiadającemu warstwie z powtórzonym wartościowaniem. Powstaje nam cykl i kończymy algorytm. \newline Po skończonej liczbie kroków algorytm się zakończy czyli nasz skonstruowany automat, będzie skończony, niedeterministyczny i będzie rozpoznawał $L'$. \newline 
Wnioskujemy, że $L'$ musi być regularne, czyli klasa języków regularnych faktycznie jest zamknięta ze względu na operację EvenLen. \newline \newline
Mały przykład: \newline 

Automat $A$


\begin{center}
        \begin{tikzpicture}[shorten >=1pt,node distance=3cm,on grid,auto]
            \node (q_1) [state, initial] at (3,0) {$q_1$};
            \node (q_2) [state, accepting] at (5,0) {$q_2$};
            \path[->] (q_1) edge node {a} (q_2)
	    (q_1) edge [loop above] node {b} ()
	    (q_2) edge [bend left] node {b} (q_1);
        \end{tikzpicture}
\end{center}

Graf pomocniczy 

\begin{center}
        \begin{tikzpicture}[shorten >=1pt,node distance=3cm,on grid,auto]
		\node (q_10) [state] at (3,0) {$q_{1,0}$};
		\node (q_20) [state, accepting] at (5,0) {$q_{2,0}$};
		\node (q_11) [state] at (3,-2) {$q_{1,1}$};
		\node (q_21) [state, accepting] at (5,-2) {$q_{2,1}$};
		\node (q_12) [state] at (3,-4) {$q_{1,2}$};
		\node (q_22) [state, accepting] at (5,-4) {$q_{2,2}$};
		\node (q_13) [state] at (3,-6) {$q_{1,3}$};
		\node (q_23) [state, accepting] at (5,-6) {$q_{2,3}$};
		\node[draw=none] at (8, 0.75) {Wartościowanie:};
		\node[draw=none] at (7.75, 0) {1, 0};
		\node[draw=none] at (7.75, -2) {1, 1};
		\node[draw=none] at (7.75, -4) {0, 1};
		\node[draw=none] at (7.75, -6) {\shortstack{1, 0 \\ pierwsze powtórzenie}};


		\node[state, draw=none] (q_14) at (3,-8) {$\cdots$};
		\node[state, draw=none] (q_24) at (5,-8) {$\cdots$};
            \path[->] 	(q_10) edge node {} (q_21)
	    		(q_10) edge node {} (q_11)
	    		(q_20) edge node {} (q_11)

			(q_11) edge node {} (q_22)
	    		(q_11) edge node {} (q_12)
	    		(q_21) edge node {} (q_12)

			(q_12) edge node {} (q_23)
	    		(q_12) edge node {} (q_13)
	    		(q_22) edge node {} (q_13)

			(q_13) edge node {} (q_24)
	    		(q_13) edge node {} (q_14)
	    		(q_23) edge node {} (q_14);
        \end{tikzpicture}
\end{center}

Skonstruowany automat \newline

\begin{center}
        \begin{tikzpicture}[shorten >=1pt,node distance=3cm,on grid,auto]
            \node (q_1) [state, initial] at (3,0) {$1,0$};
            \node (q_2) [state, accepting] at (5,0) {$1,1$};
	    \node (q_3) [state] at (7,0) {$0,1$};
	    \path[->] (q_1) edge node {} (q_2)
		(q_2) edge node {} (q_3)
		(q_3) edge [bend left] node {}(q_1);
        \end{tikzpicture}
\end{center}



\end{document}
