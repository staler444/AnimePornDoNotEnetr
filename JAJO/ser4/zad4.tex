\documentclass{article}
\usepackage[T1]{fontenc}
\usepackage{polski}
\usepackage[polish]{babel}
\usepackage[utf8x]{inputenc}
\usepackage{fontspec}
\usepackage{mathtools}
\usepackage{amssymb}
\usepackage[hidelinks]{hyperref}
\usepackage{amsmath,suetterl,graphicx,mathrsfs}
\usepackage[a4paper, total={6.2in, 10in}]{geometry}
\usepackage[skip=4pt plus1pt, indent=0pt]{parskip}

\title{JAO Praca domowa}
\author{Bartosz Kucypera}
\date{\today}

\begin{document}

\maketitle

\section*{Zadanie 4}
Dla danej formuły zdaniowej $\varphi$ w postaci CNF i wartościowania $v$ określamy score$(\varphi, v)$ jako maksymalną liczbę $k$ taką, że w każdej klauzuli formuły $\varphi$ jest conajmniej $k$ literałów, które są prawdziwe przy wartościowaniu $v$. \newline
Udowodnić, że problem jest NP-zupełny:
\subsection*{Dane:}
Formuła $\varphi$.
\subsection*{Pytanie:}
Czy istnieje wartościowanie $v$ takie, że $3\le$ score$(\varphi, v) \le 5$.

\subsection*{}
sprowadzić* - sprowadzić w czasie wielomianowym \newline \newline 
Problem jest NP-zupełny jeśli: jest NP, każdy inny problem NP możemy do niego sprowadzić*. \newline
Nasz problem oczywiście jest NP, bo dla każdej formuły i wartościowania możemy w liniowym czasie wyliczyć score i sprawdzić czy faktycznie zachodzi $3 \le score \le 5$. \newline \newline
Zauważmy, że sprowadzanie* jednego problemu do drugiego to relacja przechodnia. Jeśli jesteśmy w stanie problem A sprowadzić* do B, i B do C, to jesteśmy w stanie sprowadzić* A do C. \newline 
Skoro do problemów NP-zupełnych jesteśmy w stanie sprowadzić* wszystkie innne problemy NP, to jeśli jakiś problem NP-zupełny sporwadzimy* do naszego problemu, to z przechodniości sprowadzania*, każdy problem NP będzie można sprowadzić* do naszego.\newline

Sprowadźmy* w takim razie problem 3CNFSAT do naszego problemu. \newline \newline
literał - zmienna lub negacja zmiennej \newline
klauzula - alternatywa literałów \newline
formuła - koniunkcja klauzul \newline

3CNFSAT to problem, sparwdzenia spełnialności formuły w postaci CNF, gdzie każda klauzula ma do 3 literałów.

Niech $\varphi$ formułą takiej postaci. \newline
Przekształćmy formułę $\varphi$ do $\varphi'$ w następujący sposób.
Niech $\alpha$ i $\beta$ nowymi zmiennymi niewystępującymi jeszcze w $\varphi$. Do każdej klauzuli
$\varphi$ dodajemy literały $\alpha$, $\beta$, $\neg \alpha$, $\neg \beta$. \newline \newline
Formuła $\varphi'$ ma następujące własności: 
/* dla danego wartościowania */ \newline \newline
Każda klauzlua $\varphi'$ ma co najwyżej pięć prawdziwych literałów. (z dodanych 4 literałów, zawsze 2 są prawdziwe a starych było nie więcej niż 3, czyli prawdziwych mamy nie więcej niż 5). \newline \newline
Każda klauzula $\varphi'$ ma co najmniej dwa prawdziwe literały. (analogicznie jak powyżej)\newline 
Klauzula $\varphi'$ ma dwa prawdziwe literały, wtedy i tylko wtedy kiedy oryginalna klauzula z $\varphi$ była niespełniona (z czterech dodanych przez nas literałów zawsze dwa są prawdziwe, jeśli więc cała klauzula ma tylko dwa prawdziwe literały, to znaczy że wszystkie stare, z klauzuli z $\varphi$ są fałszywe).

Zauważmy, że istnienie wartościowania $v'$, takiego, że $3 \le$ score$(\varphi', v') \le 5$
jest równoważne spełnialności formuły $\varphi$. \newline

\subsection*{Istnienie $v' \to$ Spełnialność $\varphi$}
Skoro przy wartościowaniu $v'$ formuła $\varphi'$ ma score $\ge3$ to znaczy, że w każdej klauzli przynajmniej jeden z oryginalnych literałów z klauzuli z $\varphi$ jest prawdziwy, czyli jeśli z wartościowania $v'$ usuniemy $\alpha$ i $\beta$ to otrzymane wartościowanie $v$ będzie spełniać formułę $\varphi$.

\subsection*{Spełnialość $\varphi \to$ Istnienie $v'$}
Niech $v$ będzie wartościowaniem spełniającym $\varphi$. Wystarczy, że rozszerzymy je o $\alpha$ i $\beta$ z dowolnymi wartościami, i każda klauzula z $\varphi'$ bedzie miała score$\ge3$.
Z pośród czterech dodanych literałów, zawsze dwa są prawdziwe a skoro $v$ spełniało $\varphi$ to w każdej klauzuli $\varphi'$ przynajmniej jeden z oryginalnych literałów z $\varphi$ jest prawdziwy, czyli score jest przynajmniej 3, (nie może być większy od 5) więc zachodzi $3\le$ score$(\varphi', v)\le5$.

\subsection*{Synteza}
Skoro nasz problem jest w NP, oraz potrafimy 3CNFSAT przekształcić do naszego problemu w czasie wielomianowym, (konstrukcje $\varphi'$ z $\varphi$ wykonuje się w czasie liniowym) to nasz problem jest NP-zupełny.




\end{document}:
