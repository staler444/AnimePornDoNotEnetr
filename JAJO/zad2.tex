\documentclass{article}
\usepackage[T1]{fontenc}
\usepackage{polski}
\usepackage[polish]{babel}
\usepackage[utf8x]{inputenc}
\usepackage{fontspec}
\usepackage{mathtools}
\usepackage{amssymb}
\usepackage[hidelinks]{hyperref}
\usepackage{amsmath,suetterl,graphicx,mathrsfs}
\usepackage[a4paper, total={6.2in, 10in}]{geometry}
\usepackage[skip=4pt plus1pt, indent=0pt]{parskip}

\title{JAO praca domowa}
\author{Bartosz Kucypera}
\date{\today}

\begin{document}

\maketitle

\section*{Zadanie 1.2} 

Dla danego alfabetu $A$ oraz języka $L \subseteq A^* $ zdefinujmy SquareLen($L$) jako
$$\{w \in \{1\}^* | \mbox{ liczba słów długości } |w| \mbox{ w $L$ jest kwadratem liczby naturalnej} \}$$
Wykaż, że klasa języków regularnych nie jest zamknięta ze względu na operację SquareLen. \newline
\newline
Żeby pokazać, że klasa języków regularnych nie jest zamknięta ze względu na operację SquareLen, wystarczy, że znjadziemy język regularny który operacja SquareLen przeprowadzi na język nie-regualrny.
\newline 
Niech $A = \{a, b\}$, oraz niech $L \subseteq A^*$ opisane wyrażeniem regularnym $aa^*b^*$. \newline
Zauważmy, że dla każdego $n > 0$ istnieje dokładnie $n$ słów długości $n$ należących do $L$. \newline
Niech $L' = $ SquareLen($L$). Do $L'$ należą słowa złożne z samych jedynek o długościach kwadratów kolejnych liczb naturalnych. \newline 
Wystarczy pokazać, że $L'$ nie jest jeżykiem regularnym. Skorzystajmy, więc z fatu, iż zbiór długości słów języka regularnego, jest semiliniowy. Zbiór $A \subseteq \mathbb{N}$ jest semiliniowy, jeśli $ \exists d,c \in \mathbb{N} $, ($d>0$) takie, że dla $ \forall x \in A$ jeśli $x > c$ to $x+d \in A$. \newline 
Zbiór długości słów $L'$ nie spełnia tej definicji. \newline 
Różnica między $n+1$-wszym a $n$-tym elemntem zbioru wynosi $(n+1)^2 - n^2 = 2n+1$, oczywiście więc nie może istnieć, taka stała $c$ (odległości pomiędzy kolejnymi elementami dążą do nieskończoności).



\end{document}
