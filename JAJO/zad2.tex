\documentclass{article}
\usepackage[T1]{fontenc}
\usepackage{polski}
\usepackage[polish]{babel}
\usepackage[utf8x]{inputenc}
\usepackage{fontspec}
\usepackage{mathtools}
\usepackage{amssymb}
\usepackage[hidelinks]{hyperref}
\usepackage{amsmath,suetterl,graphicx,mathrsfs}
\usepackage[a4paper, total={6.2in, 10in}]{geometry}
\usepackage[skip=4pt plus1pt, indent=0pt]{parskip}

\title{JAO praca domowa}
\author{Bartosz Kucypera}
\date{\today}

\begin{document}

\maketitle

\section*{Zadanie 1.2} 

Dla danego alfabetu $A$ oraz języka $L \subseteq A^* $ zdefinujmy SquareLen($L$) jako
$$\{w \in \{1\}^* | \mbox{ liczba słów długości } |w| \mbox{ w $L$ jest kwadratem liczby naturalnej} \}$$
Wykaż, że klasa języków regularnych nie jest zamknięta ze względu na operację SquareLen. \newline
\newline
Żeby pokazać, że klasa języków regularnych nie jest zamknięta ze względu na operację SquareLen, wystarczy, że znjadziemy język regularny który operacja SquareLen przeprowadzi na język nie-regualrny.
\newline 
Niech $A = \{a, b\}$, oraz niech $L \subseteq A^*$ opisany wyrażeniem regularnym $aa^*b^*$. \newline
Zauważmy, że dla każdego $n > 0$ istnieje dokładnie $n$ słów długości $n$ należących do $L$. \newline
Niech $L' = $ SquareLen($L$). Do $L'$ należą słowa złożne z samych jedynek o długościach kwadratów kolejnych liczb naturalnych. \newline 
Wystarczy pokazać, że $L'$ nie jest językiem regularnym. Skorzystajmy, więc z Lematu o pompowaniu dla języków regularnych. \newline 
Załóżmy, że $L'$ jest językiem regularnym. Istnieje więc takie $n_0$, że $ \forall w \in L'$, jeśli $|w| \ge n_0$ to istnieje podział $w$ na podsłowa $x, y, z$ takie, że $$w = xyz$$ $$y \neq \epsilon$$ $$|xy| \le n_0$$ $$\forall k \ge 0, xy^kz \in L$$ \newline
Weźmy, więc takie $w_1$, że $|w_1| \ge n_0$. Z lematu o pompowaniu wynika, że istnieje takie $c > 0$ ($c = |y|, y$ z lematu), że $\forall k \in \mathbb{N}$ istnieją słowa długości $|w_1| + k*c$ należące do $L'$. \newline Niech $x_k = \sqrt{|w_1| + k*c}$ ($x_k \in \mathbb{N}$ dzięki konstrukcji $L'$). Musi zachodzić 
$$(x_k+1)^2 - x_k^2 = 2x_k + 1 \le c$$ Różnica kolejnych długości słów z $L'$ musi być nie większa niż $c$, bo istnieje w $L'$ słowo długości $x_k^2 + c$ (z konstrukcji $L'$ wiemy, że jeśli istnieje w $L'$ słowo długości $n^2$ to kolejną liczbą naturalną dla której istnieje w $L'$ słowo mające długość równą niej, jest $(n+1)^2$). Czyli $\forall k \in \mathbb{N}, 2x_k + 1 \le c$. Takie $c$ oczywiście nie istnieje, bo $\lim_{k\to\infty} 2x_k+1 = \infty $. Wnioskujemy nie wprost, że $L'$ nie jest regularne, czyli klasa języków regularnych nie jest zamknięta względem operacji SquareLen.



\end{document}
