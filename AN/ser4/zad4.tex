\documentclass{article}
\usepackage[T1]{fontenc}
\usepackage{polski}
\usepackage[polish]{babel}
\usepackage[utf8x]{inputenc}
\usepackage{fontspec}
\usepackage{mathtools}
\usepackage{amssymb}
\usepackage[hidelinks]{hyperref}
\usepackage{amsmath,suetterl,graphicx,mathrsfs}
\usepackage[a4paper, total={7in, 10in}]{geometry}
\usepackage[skip=4pt plus1pt, indent=0pt]{parskip}

\title{Analiza seria 4}
\author{Bartosz Kucypera, bk439964}
\date{\today}

\begin{document}

\maketitle

\section*{Zadanie 4}
Proszę wyznaczyć zbiór punktów zbieżności szeregu potęgowego
$$\sum_{n=1}^{\infty} \frac{(-1)^n7^{n^3}}{\ln(n)}(\sinh x)^{2n^3}$$
Niech $t = 7(\sinh x)^2$. Nasz szereg zapisuje się teraz jako:

$$ \sum_{n=1}^{\infty}\frac{(-1)^n}{\ln(n)} t^{n^3}$$

Zauważmy, że $t \ge 0$.\newline 
\subsection*{1) $t > 1$}
Jeśli $t>1$ to szereg oczywiście rozbieżny, bo nie spełnia warunku koniecznego.
$$ \left| \frac{(-1)^n}{\ln(n)}t^{n^3} \right| \ge \left| \frac{t^n}{\ln(n)} \right| $$
a $\displaystyle \lim_{n \to \infty} \frac{t^n}{\ln(n)} = \infty$ dla $t>1$.

\subsection*{2) $t < 1$}
Jeśli $t < 1$ to, badamy zbieżność bezwzględną. Zachodzi następująca nierówność:

$$ \left| \frac{(-1)^n}{\ln(n)}t^{n^3} \right| \le \left| \frac{t^n}{\ln(n)} \right| $$

a szereg
$$\sum_{n=1}^{\infty} \left| \frac{t^n}{\ln(x)} \right| $$ szeregiem potęgowym o promieniu zbieżności $R = 1$ (szeregiem potęgowym zmiennej $t$), czyli zbieżny bezwzględnie dla $t<1$, czyli wyjściowy szereg też zbieżny bezwzględnie dla $t<1$ z kryterium porównawczego.

\subsection*{3) $t = 1$}
Jeśli $t =1$ to otrzymujemy szereg:
$$ \sum_{n=1}^{\infty} \frac{(-1)^n}{\ln(n)}$$
który natychmiast zbieżny z kryterium Leibniz'a.

\newpage

\subsection*{Odzyskanie zbioru punktów zbierzności}
Musimy jeszcze tylko odzyskać, dla jakich $x$ szereg będzie zbieżny. Szereg zbieżny dla $x$ spełniającyh
$$ 7\sinh(x)^2 \le 1$$
czyli 
$$|\sinh(x)| \le \frac{1}{\sqrt{7}}$$
czyli
$$\sinh(x) \in \left[ \frac{-1}{\sqrt{7}}, \frac{1}{\sqrt{7}} \right]$$
czyli*
$$ x \in \left[-\sinh^{-1}\left(\frac{1}{\sqrt{7}}\right), \sinh^{-1}\left(\frac{1}{\sqrt{7}}\right) \right] $$
*bo $\sinh$ ściśle rosnący, oraz $\sinh^{-1}(-x) = -\sinh^{-1}(x)$ \newline
czyli 
$$ x \in \left[\frac{\ln7}{2}-\ln(2\sqrt{2}+1), \ln(2\sqrt{2}+1)-\frac{\ln7}{2} \right]$$

\end{document}
