\documentclass{article}
\usepackage[T1]{fontenc}
\usepackage{polski}
\usepackage[polish]{babel}
\usepackage[utf8x]{inputenc}
\usepackage{fontspec}
\usepackage{mathtools}
\usepackage{amssymb}
\usepackage[hidelinks]{hyperref}
\usepackage{amsmath,suetterl,graphicx,mathrsfs}
\usepackage[a4paper, total={7in, 12in}]{geometry}
\usepackage[skip=4pt plus1pt, indent=0pt]{parskip}

\title{Analiza seria 5}
\author{Bartosz Kucypera, bk439964}
\date{\today}

\begin{document}

\maketitle

\section*{Zadanie 5}
Niech $0 < a < b < \infty$ oraz $f:[0,1] \to \mathbb{R}$ będzie funkcją ciągłą. Obliczyć granicę
$$\lim_{\epsilon \to 0^+} \int_{a\epsilon}^{b\epsilon}\frac{f(x)}{x}dx$$
Zauważmy, że dla dowolnego $\epsilon_1>0$, możemy dobrać tak mały $\epsilon_0>0$, że $\forall \epsilon, 0<\epsilon<\epsilon_0$ zachodzi:
$$\int_{a\epsilon}^{b\epsilon}\frac{f(0)-\epsilon_1}{x}dx \le 
\int_{a\epsilon}^{b\epsilon}\frac{f(x)}{x}dx \le 
\int_{a\epsilon}^{b\epsilon}\frac{f(0)+\epsilon_1}{x}dx,$$
bo $f$ ciągła prawostronnie w 0, czyli jeśli $a\epsilon$ i $b\epsilon$ dostatecznie małe, to $\forall x\in[a\epsilon, b\epsilon], |f(x)-f(0)| < \epsilon_1$.\\
Teraz zauważmy, że:
$$\int_{a\epsilon}^{b\epsilon}\frac{f(0)\pm\epsilon_1}{x}dx = (f(0)\pm\epsilon_1)\int_{a\epsilon}^{b\epsilon}\frac{dx}{x} = (f(0)\pm\epsilon_1)(\ln(b\epsilon) - \ln(a\epsilon)) = (f(0)\pm\epsilon_1)(\ln(b)-\ln(a)) $$
Czyli teraz dla dowolonego $\epsilon_2$ możemy dobrać $\epsilon_1 = \frac{\epsilon_2}{\ln(b)-\ln(a)}$ i do tego $\epsilon_1$ taki $\epsilon_0$, że $\forall \epsilon, 0 < \epsilon < \epsilon_0$, zachodzi:
$$\left| \int_{a\epsilon}^{b\epsilon}\frac{f(x)}{x}dx - \int_{a\epsilon}^{b\epsilon} \frac{f(0)}{x}dx\right| = 
\left| \int_{a\epsilon}^{b\epsilon}\frac{f(x)}{x}dx - f(0)(\ln(b)-\ln(a)) \right| \le 
\epsilon_1(\ln(b)-\ln(a)) = \epsilon_2$$
Czyli,
$$\lim_{\epsilon\to0^+} \int_{a\epsilon}^{b\epsilon}\frac{f(x)}{x}dx = f(0)\ln\left(\frac{b}{a}\right),$$
z definicji.
\end{document}
