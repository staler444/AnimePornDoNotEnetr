\documentclass{article}
\usepackage[T1]{fontenc}
\usepackage{polski}
\usepackage[polish]{babel}
\usepackage[utf8x]{inputenc}
\usepackage{fontspec}
\usepackage{mathtools}
\usepackage{amssymb}
\usepackage[hidelinks]{hyperref}
\usepackage{amsmath,suetterl,graphicx,mathrsfs}
\usepackage[a4paper, total={7in, 12in}]{geometry}
\usepackage[skip=4pt plus1pt, indent=0pt]{parskip}

\title{Analiza seria 5}
\author{Bartosz Kucypera, bk439964}
\date{\today}

\begin{document}

\maketitle

\section*{Zadanie 2} 
Proszę obliczyć całki i formalnie uzasadnić otrzymany wynik

$$ a) \int_{-1}^1 \frac{x^7-3x^5 +7x^3-x}{\cos^2(x)}dx \mbox{ oraz } 
b) \int_{-\pi/3}^{\pi/3}x^{10}\sin ^9(x)dx $$

\section*{a)}
Niech $f:[-1,1] \to \mathbb{R}$ dana wzorem $f(x) = \frac{x^7-3x^5+7x^3-x}{\cos^2(x)}$. \\
Zauważmy, że $f(x) = -f(-x)$, bo $\cos^2(-x) = \cos^2(x)$ a licznik jest wielomianem o niezerowych współczynnikach tylko przy nieparzystych potęgach $x$. \\
Rozbijamy całkę na dwie części:

$$\int_{-1}^{1}f(x)dx = \int_{-1}^0f(x)dx + \int_0^1f(x)dx$$
Podstawiamy $t = -x$ w pierwszej całce:

$$\int_{-1}^0f(x)dx + \int_0^1f(x)dx = -\int_1^0f(-t)dt + \int_0^1f(x)dx = -\int_0^1-f(-t)dt + \int_0^1f(x)dx$$ 
Korzystamy z tego, że $-f(-t) = f(t)$:
$$-\int_0^1-f(-t)dt + \int_0^1f(x)dx = -\int_0^1f(t)dt + \int_0^1f(x)dx = 0$$

\section*{b)}
Robimy dokładnie to samo co w a). \\
Niech $f:[-\pi/3, \pi/3] \to \mathbb{R}$ dana wzorem $f(x) = x^{10}\sin^9(x)$. \\
Zauważmy, że $f(x) = -f(-x)$, bo $(-x)^{10} = x^{10}$ a $\sin$ jest nieparzysty, więc $\sin^9$ też. \\
Rozbijamy całkę:
$$ \int_{-\pi/3}^{\pi/3}f(x)dx = \int_{-\pi/3}^{0}f(x)dx + \int_{0}^{\pi/3}f(x)dx$$
Podstawaimy $t=-x$ w pierwszej całce:
$$\int_{-\pi/3}^{0}f(x)dx + \int_{0}^{\pi/3}f(x)dx = -\int_0^{\pi/3}-f(-t)dt + \int_0^{\pi/3}f(x)dx$$
Korzystamy z tego, że $-f(-x) = f(x)$:
$$-\int_0^{\pi/3}-f(-t)dt + \int_0^{\pi/3}f(x)dx = -\int_0^{\pi/3}f(t)dt + \int_0^{\pi/3}f(x)dx = 0$$

\end{document}
