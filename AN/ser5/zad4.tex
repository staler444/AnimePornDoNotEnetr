\documentclass{article}
\usepackage[T1]{fontenc}
\usepackage{polski}
\usepackage[polish]{babel}
\usepackage[utf8x]{inputenc}
\usepackage{fontspec}
\usepackage{mathtools}
\usepackage{amssymb}
\usepackage[hidelinks]{hyperref}
\usepackage{amsmath,suetterl,graphicx,mathrsfs}
\usepackage[a4paper, total={7in, 12in}]{geometry}
\usepackage[skip=4pt plus1pt, indent=0pt]{parskip}

\title{Analiza seria 5}
\author{Bartosz Kucypera, bk439964}
\date{\today}

\begin{document}

\maketitle

\section*{Zadanie 4}
Niech $f : [0,1] \to (0,\infty)$ będzie funkcją ciągłą. Obliczyć całkę
$$\int_0^1\frac{f(x)}{f(x)+f(1-x)}dx$$
Zauważmy, że:

$$\int_0^1\frac{f(x)}{f(x)+f(1-x)}dx = \int_0^1\frac{f(x) + f(1-x) - f(1-x)}{f(x) + f(1-x)}dx
= \int_0^1dx - \int_0^1\frac{f(1-x)}{f(x)+f(1-x)}dx$$
Podstawaiamy do prawej całki $t=1-x$:
$$\int_0^1dx - \int_0^1\frac{f(1-x)}{f(x)+f(1-x)}dx = 1 - \int_1^0\frac{-f(t)}{f(1-t)+f(t)}dt= 1 - \int_0^1\frac{f(t)}{f(1-t)+f(t)}dt$$
Czyli po przeniesieniu całki z powrotem na lewą stronę równości, otrzymujemy:
$$2\int_0^1\frac{f(x)}{f(x)+f(1-x)}dx = 1$$
czyli
$$\int_0^1\frac{f(x)}{f(x)+f(1-x)}dx = \frac{1}{2}$$

\end{document}
