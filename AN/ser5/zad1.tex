\documentclass{article}
\usepackage[T1]{fontenc}
\usepackage{polski}
\usepackage[polish]{babel}
\usepackage[utf8x]{inputenc}
\usepackage{fontspec}
\usepackage{mathtools}
\usepackage{amssymb}
\usepackage[hidelinks]{hyperref}
\usepackage{amsmath,suetterl,graphicx,mathrsfs}
\usepackage[a4paper, total={7in, 10in}]{geometry}
\usepackage[skip=4pt plus1pt, indent=0pt]{parskip}

\title{Analiza seria 5}
\author{Bartosz Kucypera, bk439964}
\date{\today}

\begin{document}

\maketitle

\section*{Zadanie 1}

Niech $\displaystyle f : \mathbb{R} \to \mathbb{R} $ będzie funkcją ciągłą. Udowodnić, że

$$\lim_{n \to \infty} \int_{0}^{2\pi} f(x)\sin(nx)dx = 0$$

\subsection*{Lemacik}
$$\lim_{n\to \infty} \int_a^bc \cdot \sin(nx)dx = 0$$
dla dowlonych $a, b, c, a<b,$ rzeczywistych.

Jeśli $b' - a'$ wielokrotnością okresu $\sin(nx)$ to oczywiście
$$ \int_{a'}^{b'} c \cdot \sin(nx)dx = 0, \mbox{ czyli też } \lim_{n\to\infty}\int_{a'}^{b'}c \cdot \sin(nx)dx = 0$$
Zauważmy, że
$$\int_a^bc \cdot \sin(nx)dx = \int_{a}^{d} c \cdot \sin(nx)dx + \int_{d}^{b}c \cdot \sin(nx)dx $$
gdzie $d-a$ największą wielokrotnością okresu $\sin(nx)$, mniejszą od $b-a$.
Zachodzi, więc następujący ciąg równości i nierówności
$$\int_a^b c \cdot \sin(nx)dx = \int_a^d c \cdot \sin(nx)dx + \int_d^b c \cdot \sin(nx)dx \le (b-d)\cdot c \le \frac{2\pi}{n}c $$
bo $b-d$ mniejsze równe od okresu $\sin(nx)$, czyli $\frac{2\pi}{n}$. \newline

Mamy więc, 

$$\lim_{n\to \infty} \left| \int_a^bc\cdot\sin(nx)dx - 0 \right| \le
\lim_{n\to\infty} \left| \frac{2\pi}{n}c \right| = 0 $$

Czyli faktycznie 
$$\lim_{n\to\infty}\int_a^bc\cdot \sin(nx)dx = 0$$

\newpage

\subsection*{Funkcja pomocnicza $g_{\epsilon}$}


Niech funkcja $g_{\epsilon}$, $g_{\epsilon}: \mathbb{R} \to \mathbb{R}$, zdefiniowana następująco $\forall \epsilon >0$, (poniżej zakładamy, że $\epsilon$ jest już ustalony):

Niech $\delta$ taka, że $\forall x,y \in [0,2\pi]$ jeśli $|x-y| < \delta$ to $|f(x) - f(y)| < \frac{\epsilon}{2\pi} \cdot \frac{1}{2}$. Taka $\delta$ istnie, bo $f(x)$ jest ciągła, przedział
$[0, 2\pi]$ domknięty, czyli $f(x)$ ciągła jednostajnie na $[0, 2\pi]$.

Teraz dzielimy $[0, 2\pi]$ na przedziały długości $\delta$ (jak $\delta$ nie dzieli $2\pi$, to ostatni krótszy) i na każdym przedziale postaci $[k\delta, (k+1)\delta)$, (lub jakiś ostatni postaci $[k\delta, 2\pi]$), ustalamy $g_{\epsilon}(x) = f(k\delta)$ dla $x\in [k\delta, (k+1)\delta)$, (analogicznie dla tego ostatniego).

Dzięki takiej konstrukcji, $g_{\epsilon}$ ma następujące, przydatne, własności:

\subsection*{1)}

$$ \lim_{n\to\infty}\int_0^{2\pi}g_{\epsilon}(x)\sin(nx)dx = 0$$
dla ustalonego $\epsilon$. $g_{\epsilon}$ stała na stałej liczbie przedzaiłów, do każdego przedziału przykładamy $Lemacik$.
\subsection*{2)}
$$ \left| \int_0^{2\pi}|f-g_{\epsilon}|(x)\sin(nx)dx \right| \le \frac{\epsilon}{2\pi}\cdot\frac{1}{2} \cdot2\pi = \frac{\epsilon}{2},$$
bo z jednostajnej ciągłości $f$, na każdym przedziale $I=[k\delta, (k+1)\delta)$, 
$$\forall x \in I,\;\; |f(x)-f(k\delta)| < \frac{\epsilon}{2\pi}\cdot\frac{1}{2}$$ 
a skoro dla $x\in I, g_{\epsilon}(x) = f(k\delta)$ to
$$\forall x \in I, \;\; |f(x) - g_{\epsilon}(x)| < \frac{\epsilon}{2\pi}\cdot\frac{1}{2},$$
powtarzamy to rozumowanie dla każdego przedzaiłu gdzie $g_{\epsilon}$ stałe i otrzymujemy, że
$$ \forall x \in [0, 2\pi], |f(x)-g_{\epsilon}(x)| < \frac{\epsilon}{2\pi}\cdot\frac{1}{2}$$

\subsection*{Rozwiązanie}
Korzystając z własności $1)$ i $2)$, mamy \newline

$\forall \epsilon > 0, \exists N$ takie, że  $\forall n > N$ zachodzi

$$ \left| \int_0^{2\pi}f(x)\sin(nx)dx - 0 \right| =
\left| \int_0^{2\pi} |f-g_{\epsilon}|(x)\sin(nx)dx + \int_0^{2\pi}g_{\epsilon}(x)\sin(nx)dx \right|
\le $$ 
$$ \le \left| \int_0^{2\pi} |f-g_{\epsilon}|(x)\sin(nx)dx \right| + \left| \int_0^{2\pi}g_{\epsilon}(x)\sin(nx)dx \right| = \frac{\epsilon}{2} + \frac{\epsilon}{2} = \epsilon $$
czyli faktycznie
$$\lim_{n\to\infty}\int_0^{2\pi}f(x)\sin(nx)dx = 0,$$
z definicji.


\end{document}
