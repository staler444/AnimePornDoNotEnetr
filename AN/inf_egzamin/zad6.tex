\documentclass{article}
\usepackage[T1]{fontenc}
\usepackage{polski}
\usepackage[polish]{babel}
\usepackage[utf8x]{inputenc}
\usepackage{fontspec}
\usepackage{mathtools}
\usepackage{amssymb}
\usepackage[hidelinks]{hyperref}
\usepackage{amsmath,suetterl,graphicx,mathrsfs}
\usepackage[a4paper, total={7in, 12in}]{geometry}
\usepackage[skip=4pt plus1pt, indent=0pt]{parskip}

\title{Egzamin AN. inf. II}
\author{Bartosz Kucypera, bk439964}
\date{\today}

\begin{document}

\maketitle

Rozwiązanie przygotowałem samodzielnie, ze świadomością, iż etyczne zdobywanie zaliczeń jest, zgodnie z Regulaminem Studiów, obowiązkiem studentek i studentów UW.

\section*{Zadanie 6}
Obliczyć całkę

$$ \int_D \cos(x^2+y^2)\lambda_2(x, y)$$
gdzie 
$$ D =\{(x,y)\in \mathbb{R}^2 | x \in[-1, 1], -\sqrt{1-x^2} \le y \le 0\}, $$
a $\lambda_2$ jest miarą Lebesgue'a na płaszczyźnie.

Najpier zamieńmy zmienne na biegunowe. Niech 
$$ x = r\cos\varphi, y = r\sin\varphi$$
Teraz
$$ -\sqrt{1-x^2} \le y \le 0$$ 
$$ -\sqrt{1-r^2\cos^2\varphi} \le  r\sin\varphi \le 0$$
podnosimy do kwadratu, nierówności się odwracają
$$ 1 -r^2\cos^2\varphi \ge r^2\sin^2\varphi \ge 0 $$
Czyli mamy 
$$ 1 \ge r^2(\sin^2\varphi + \cos^2\varphi) = r^2 $$
czyli
$$ 1 \ge r \ge 0 $$
Dodatkowo skoro $r\sin\varphi \le 0$ to $\varphi \in [-\pi, 0]$. \newline

Całka po zamianie

$$ \int_{D'}\cos(r^2\cos^2\varphi + r^2\sin^2\varphi) \cdot r \lambda_2(r, \varphi) $$
(domnażamy wyznaczik pochodnej podsawienia, czyli $r$ dla podstawienia zmiennych biegunowych).
$$D' = \{(r\cos\varphi, r\sin\varphi) | r \in [0, 1], \varphi \in [-\pi, 0]\}$$
$D'$ jest zbiorem zwartym, funkcja podcałkowa ograniczona, czyli całka napewno zbieżna. Możemy skorzystać z twierdzenia Fubiniego.
$$ \int_{D'}\cos(r^2) \cdot r \lambda_2(r, \varphi) = 
\int_{[-\pi, 0]} \left( \int_{[0, 1]}\cos(r^2)r\lambda(r) \right)\lambda(\varphi) = 
\int_{-\pi}^0 \left(\int_0^1\cos(r^2)r dr\right)d\varphi
$$ 
Wewnętrzną całkę liczymy z podstawienia $t = r^2$

$$ \frac{1}{2}\int_0^1\cos(t)dt = \frac{\sin1}{2}$$

$$ \int_{-\pi}^0\frac{\sin1}{2}d\varphi = \frac{\pi}{2}\sin1 $$



\end{document}
