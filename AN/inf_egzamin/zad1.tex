\documentclass{article}
\usepackage[T1]{fontenc}
\usepackage{polski}
\usepackage[polish]{babel}
\usepackage[utf8x]{inputenc}
\usepackage{fontspec}
\usepackage{mathtools}
\usepackage{amssymb}
\usepackage[hidelinks]{hyperref}
\usepackage{amsmath,suetterl,graphicx,mathrsfs}
\usepackage[a4paper, total={7in, 10in}]{geometry}
\usepackage[skip=4pt plus1pt, indent=0pt]{parskip}

\title{Egzamin AN. inf. II}
\author{Bartosz Kucypera, bk439964}
\date{\today}

\begin{document}

\maketitle

Rozwiązanie przygotowałem samodzielnie, ze świadomością, iż etyczne zdobywanie zaliczeń jest, zgodnie z Regulaminem Studiów, obowiązkiem studentek i studentów UW.

\section*{Zadanie 1}
Niech 
$$ f(x,y) = xy -x^2y - xy^2.$$

Wyznaczyć punkty krytyczne funkcji $f$, leżące w pierwszej ćwiartce $(x, y > 0)$. Zbadać istnienie ekstremów funkcji $f$ w tych punktach i określić ich rodzaj.

\section*{}
Funkcja $f:\mathbb{R}^2 \to \mathbb{R}$, więc zerowanie się pochodnych kierunkowych dla danego punktu jest wystarczającym warunkiem by był punktem krytycznym. Obliczmy, więc pochodne kierunkowe.

$$\frac{\partial f}{\partial x} = y- 2xy -y^2$$ 
oraz 
$$\frac{\partial f}{\partial y} = x - x^2 -2xy$$
Obliczmy punkty gdzie pochodne się zerują:

$$y-2xy -y^2 = 0, x-x^2-2xy = 0$$
czyli 
$$2xy = y-y^2, 2xy = x-x^2 $$
a skoro $x > 0, y > 0$, to $2xy > 0$, czyli na pewno musi zachodzić $0 < x, y < 1$, (jeśli $x \ge 1$ to $x-x^2 \le 0)$. \newline
Mamy 
$$ y-2xy -y^2 = x-x^2-2xy$$
czyli 
$$ y-y^2 = x-x^2$$
czyli 
$$y = x-x^2 + y^2 $$
podstawiajc do wcześniejszego równania
$$x - x^2 -2x(x-x^2 + y^2) = 0$$
$$1 - x -2(x-x^2 + y^2) = 0$$
$$ y^2 = \frac{1 - 3x - 2x^2}{2} $$
$$y = \sqrt{x^2 - \frac{3}{2}x + \frac{1}{2}}$$
\newpage
Znowu podstawiając do wcześniejszego równania
$$ x-x^2 -2x\sqrt{x^2 - \frac{3}{2}x + \frac{1}{2}} = 0$$
$$1 - x = 2 \sqrt{x^2 - \frac{3}{2}x + \frac{1}{2}}$$
$$\left(\frac{1-x}{2}\right)^2 = x^2 - \frac{3}{2}x + \frac{1}{2}$$
czyli
$$ -\frac{3}{4}x^2 + x - \frac{1}{4} = 0$$
Co ma tylko jedno rozwiązanie dla $x \in (0, 1)$, w $x = \frac{1}{3}$.
$$ x = \frac{1}{3}, y = \sqrt{\frac{1}{9} - \frac{3}{2}\cdot\frac{1}{3} + \frac{1}{2}} = \frac{1}{3} $$
Czyli jedyny punkt krytyczny w pierwszej ćwiartce to $(\frac{1}{3}, \frac{1}{3})$.
\section*{}
Teraz wykażmy, że w znalezionym punkcie krytycznym, funkcja ma ekstremum. \newline
Najpierw policzmy drugą pochodną w $a = (\frac{1}{3}, \frac{1}{3})$. 

$$Df^2(a) = 
\begin{bmatrix}
	-2\cdot\frac{1}{3} & 1 -2\cdot\frac{1}{3} -2\cdot\frac{1}{3} \\
	1-2\cdot\frac{1}{3}-2\cdot\frac{1}{3} & -2\cdot\frac{1}{3}
\end{bmatrix} =
\begin{bmatrix}
	-\frac{2}{3} & -\frac{1}{3} \\
	-\frac{1}{3} & -\frac{2}{3}
\end{bmatrix}
$$
Czyli dla dowolnego wektora $v = (v_1,v_2)$, zachodzi:

$$<D^2f(a)\cdot v, v> = -\frac{2}{3}v_1^2 - \frac{1}{3}v_1v_2 - \frac{1}{3}v_1v_2 - \frac{2}{3}v_2^2 = -\frac{1}{3}(v_1^2 + v_2^2 + (v_1+v_2)^2) < 0, \mbox{ dla } v \neq 0$$
Czyli zachodzą warunki wystarczające by $f$ miała w $a$ maksimum lokalne.

\end{document}
