\documentclass{article}
\usepackage[T1]{fontenc}
\usepackage{polski}
\usepackage[polish]{babel}
\usepackage[utf8x]{inputenc}
\usepackage{fontspec}
\usepackage{mathtools}
\usepackage{amssymb}
\usepackage[hidelinks]{hyperref}
\usepackage{amsmath,suetterl,graphicx,mathrsfs}
\usepackage[a4paper, total={7in, 12in}]{geometry}
\usepackage[skip=4pt plus1pt, indent=0pt]{parskip}

\title{Analiza seria 3}
\author{Bartosz Kucypera, bk439964}
\date{\today}

\begin{document}

\maketitle

\section*{Zadanie 5}
$$S(x) = \displaystyle \sum_{n=1}^{\infty}\frac{\cos(n\pi)\arctan(x/ \sqrt{n})}{\sqrt{n}} $$

\subsection*{Zbieżność punktowa $S$}
Zauwżmy, że: \newline 

Dla $x > 0$, $S(x) = \sum_{n=1}^{\infty} (-1)^n \frac{\arctan(x /\sqrt{n})}{\sqrt{n}}$, a skoro:
$$\left( \frac{x}{\sqrt{n} }\right) \mbox{ ściśle malejący, oraz } \arctan  \mbox{ściśle rosnący na } [0,\infty] \mbox{ to } \frac{\arctan(x /\sqrt{n})}{\sqrt{n}} \mbox{ monotonicznie zbiega do } 0.$$
Czyli z kryterium Leibniza $S$ zbieżny punktowo dla $x > 0$. \newline
Dla $x < 0$:
$$S(x) = -S(-x), \mbox{ bo } \arctan(x) = -\arctan(-x). $$
Czyli też zbieżny dla $x < 0$, bo $-x > 0$.

Dla $x = 0$, $S(0) = 0$.

\subsection*{Zbieżność jednostajna ciągu pochodnych sum częściowych}

Neich $(S_N)_{1}^{\infty}$ będzie ciągiem sum częściowych $S$.
$(S_N)$ dobrze określony i różniczkowalny na $\mathbb{R}$. \newline
Zbadajmy $S_N'$:
$$ S_N'(x) = \sum_{n=1}^{N} \left( \frac{(-1)^n \arctan(x/ \sqrt{n})}{\sqrt{n}} \right)' =
\sum_{n=1}^{N} \frac{(-1)^n}{\sqrt{n}} \cdot \frac{1}{1 + x^2 /n} \cdot \frac{1}{\sqrt{n}} =
\sum_{n=1}^{N} \frac{(-1)^n}{n+x^2}$$

$$ \sum_{n=1}^{N}(-1)^n \mbox{ ograniczony} $$
$$\frac{1}{n+x^2} \le \frac{1}{n} \mbox{ czyli z kryterium Weierstrassa, zbieżny jednostajnie do 0}$$
$(S_N')$ spełnia, więc założenia jednostajnego kryterium Dirichlet'a, czyli jest zbieżny jednostajnie do pewnej ciągłej funkcji $g$ (ciągłej, bo każdy element ciągu $(S_N')$ jest skończoną sumą funckji ciągłych, czyli jest ciągły). \newline \newline

\subsection*{Konkluzja}

Niech $[a,b] \subset \mathbb{R}$, dowolnym przedziałem zwartym. \newline

Teraz skoro $S_N' \rightrightarrows g$ na $[a,b]$, oraz ciąg $(S_N(a))$ zbieżny, możemy skożystać z twierdzenia 7.19 ze skryptu Pawła Strzeleckiego, o różniczkowaniu ciągów funkcyjnych. \newline 
Wnioskujemy, że: 
$$(S_N) \mbox{ zbieżny jednostajnie, czyli } S \mbox{ ciągła i określona na } [a,b],$$
$$ S \mbox{ różniczkowalna i } S' = g, \mbox{ na } [a,b]$$
Ciągłość $g$ mieliśmy już wcześniej. \newline
Z dowolności wyboru $[a,b]$ wnioskujemy, że $S$ klasy $C^1$ na $\mathbb{R}$, (różniczkowalność i ciągłość to własności lokalne).


\end{document}
