\documentclass{article}
\usepackage[T1]{fontenc}
\usepackage{polski}
\usepackage[polish]{babel}
\usepackage[utf8x]{inputenc}
\usepackage{fontspec}
\usepackage{mathtools}
\usepackage{amssymb}
\usepackage[hidelinks]{hyperref}
\usepackage{amsmath,suetterl,graphicx,mathrsfs}
\usepackage[a4paper, total={7in, 12in}]{geometry}
\usepackage[skip=4pt plus1pt, indent=0pt]{parskip}

\title{Analiza seria 3}
\author{Bartosz Kucypera, bk439964}
\date{\today}

\begin{document}

\maketitle

\section*{Zadanie 5}
$$S(x) = \displaystyle \sum_{n=1}^{\infty}\frac{\cos(n\pi)\arctan(x/ \sqrt{n})}{\sqrt{n}} $$
Neich $(S_N)_{1}^{\infty}$ będzie ciągiem sum częściowych $S$. 
$(S_N)$ dobrze określony i różniczkowalny na $\mathbb{R}$. \newline
Zbadajmy $S_N'$:
$$ S_N'(x) = \sum_{n=1}^{N} \left( \frac{(-1)^n \arctan(x/ \sqrt{n})}{\sqrt{n}} \right)' =
\sum_{n=1}^{N} \frac{(-1)^n}{\sqrt{n}} \cdot \frac{1}{1 + x^2 /n} \cdot \frac{1}{\sqrt{n}} =
\sum_{n=1}^{N} \frac{(-1)^n}{n+x^2}$$

$$ \sum_{n=1}^{N}(-1)^n \mbox{ ograniczony} $$
$$\frac{1}{n+x^2} \le \frac{1}{n} \mbox{ czyli z kryterium Weierstrassa, zbieżny jednostajnie do 0}$$
$(S_N')$ spełnia, więc założenia jednostajnego kryterium Dirichlet'a, czyli jest zbieżny jednostajnie do pewnej ciągłej funkcji $g$ (ciągłej, bo każdy element ciągu $(S_N')$ jest skończoną sumą funckji ciągłych, czyli jest ciągły). \newline \newline

Teraz skoro ciąg $(S_N(0))$ zbieżny (bo $S(0) = 0$), oraz $S_N' \rightrightarrows g$ możemy skożystać z twierdzenia 7.19 ze skryptu Pawła Strzeleckiego. \newline 
Wnioskujemy, że: 
$$(S_N) \mbox{ zbieżny jednostajnie, czyli } S \mbox{ ciągła i określona na } \mathbb{R},$$
$$ S \mbox{ różniczkowalna i } S' = g. $$
Ciągłość $g$ mieliśmy już wcześniej.
Czyli faktycznie $S$ klasy $C^1$.


\end{document}
