\documentclass{article}
\usepackage[T1]{fontenc}
\usepackage{polski}
\usepackage[polish]{babel}
\usepackage[utf8x]{inputenc}
\usepackage{fontspec}
\usepackage{mathtools}
\usepackage{amssymb}
\usepackage[hidelinks]{hyperref}
\usepackage{amsmath,suetterl,graphicx,mathrsfs}
\usepackage[a4paper, total={7in, 12in}]{geometry}
\usepackage[skip=4pt plus1pt, indent=0pt]{parskip}

\title{Analiza seria 3}
\author{Bartosz Kucypera, bk439964}
\date{\today}

\begin{document}

\maketitle

\section*{Zadanie 2}
$$f_n:[-1, 1] \to \mathbb{R}, f_n(x) = \frac{x}{1+n^2x^2}$$

\subsection*{Zbieżność jednostajna $f_n$}
$$ |f_n(x)| = \frac{|x|}{1+n^2|x|^2} \le^* \frac{1}{n}$$

$$*) \; \; \frac{|x|}{1+n^2|x|^2} \le \frac{1}{n} \Leftrightarrow 0 \le n^2|x|^2 - n|x| + 1, 
\Delta = -3n^2, \mbox{ czyli spełnione dla } n > 0 $$

$f_n$ punktowo zbieżny do zera oraz $|f_n - 0| < \frac{1}{n}$, czyli $f_n$ jednostajnie zbieżny z definicji.

\subsection*{Zbieżnosć jednostajna $f_n'$}

$$f_n'(x) = \frac{-n^2x^2}{(1+n^2x^2)^2} $$

Dla $x = 0$
$$f_n'(x) = 0$$
Dla $x \ne 0$
$$|f_n'(x)| = \frac{n^2x^2}{(1+n^2x^2)^2} \le \frac{1}{n^2x^2}$$
$$\displaystyle \lim_{n \to \inf } \frac{1}{n^2x^2} = 0 $$
Czyli $f_n'$ punktowo zbiega do 0. \newline \newline

Neich $x_n = \frac{1}{n}$. Zauważmy, że:
$$ f_n'(x_n) = -\frac{1}{2} $$
Czyli $f_n'$ nie jest zbieżne jednostajnie.

\subsection*{Różniczkowalność $f(x) = \displaystyle \lim_{n \to \inf} f_n(x)$}
Tak, skoro $f(x) = 0$ to $f$ jest różniczkowalne.






\end{document}
