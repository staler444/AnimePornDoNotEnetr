\documentclass{article}
\usepackage[T1]{fontenc}
\usepackage{polski}
\usepackage[polish]{babel}
\usepackage[utf8x]{inputenc}
\usepackage{fontspec}
\usepackage{mathtools}
\usepackage{amssymb}
\usepackage[hidelinks]{hyperref}
\usepackage{amsmath,suetterl,graphicx,mathrsfs}
\usepackage[a4paper, total={7in, 12in}]{geometry}
\usepackage[skip=4pt plus1pt, indent=0pt]{parskip}

\title{Analiza seria 3}
\author{Bartosz Kucypera, bk439964}
\date{\today}

\begin{document}

\maketitle

\section*{Zadanie 1} 
$$f_n : [1, \infty] \to \mathbb{R}, f_n(x) = n(\sqrt[\leftroot{0} \uproot{0} n]{x} - 1) \mbox{ dla } n \in \mathbb{N}$$

\subsection*{Zbieżność punktowa}
$$ \displaystyle \lim_{n \to \infty} n(\sqrt[n]{x} - 1) = 
\ln(x) \lim_{n \to \infty} \frac{x^\frac{1}{n} -1}{\frac{\ln(x)}{n}} = 
\ln(x) \lim_{n \to \infty} \frac{\exp(\frac{\ln(x)}{n}) - \exp(0)}{\frac{\ln(x)}{n}} = 
\ln(x) \lim_{h \to 0} \frac{\exp(h) - \exp(0)}{h}  = \ln(x)$$
Czyli $(f_n)$ punktowy zbieżny do $\ln$.
\subsection*{Zbieżność jednostajna}
Niech $x_n = n^n$. Zauważmy, że:
$$ |n(\sqrt[n]{x_n}-1) - f_n(x_n)| = |n(n-1) - n\ln(n)| = |n(n-1-\ln(n))| $$
Oczywiście $\lim_{n \to \infty} |n(n-1\ln(n))| = \infty$, więc ciąg $(f_n)$ nie spełnia jednostajnego warunku Cauchy'ego.
\subsection*{Zbieżność niemal jednostajna}
Zauważmy, że skoro $\sqrt[n]{x}$ są rosnące na $[1, \infty]$ to funkcje $f_n$ też. \newline 
Teraz, dla dowolnego $R \in \mathbb{R}, R > 1$, funkcje $f_n$ na przedziałach postaci $[1, R]$ są
rosnącę, oraz ciąg $(f_n)$ jest punktowo zbieżny do funckji ciągłej, czyli $f_n$ spełnia założenia drugiego twierdzenia Diniego. Ciag $(f_n)$ zbiega niemal jednostajnie do $\ln$.

\end{document}
