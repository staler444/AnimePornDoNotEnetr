\documentclass{article}
\usepackage[T1]{fontenc}
\usepackage{polski}
\usepackage[polish]{babel}
\usepackage[utf8x]{inputenc}
\usepackage{fontspec}
\usepackage{mathtools}
\usepackage{amssymb}
\usepackage[hidelinks]{hyperref}
\usepackage{amsmath,suetterl,graphicx,mathrsfs}
\usepackage[a4paper, total={7in, 12in}]{geometry}
\usepackage[skip=4pt plus1pt, indent=0pt]{parskip}

\title{Analiza seria 3}
\author{Bartosz Kucypera, bk439964}
\date{\today}

\begin{document}

\maketitle

\section*{Zadanie 3}
$$ \displaystyle S(x) =\sum_{n=1}^{\infty} |x|^{\sqrt{n}} $$
Dla $|x| \ge 1$ szereg nie spełnia warunku koniecznego zbieżności, czyli rozbieżny. \newline \newline
Dla $|x| < 1$, zauważmy, że:
$$\displaystyle \frac{|x|^{\sqrt{n}}}{\frac{1}{n^2}} = \exp(\sqrt{n}\ln(|x|) +2\ln(n)) = 
\exp\left(\sqrt{n}\ln(|x|) \left( 1 + \frac{2}{\ln(|x|)}\frac{\ln(n)}{\sqrt{n}} \right) \right) $$
$$ \displaystyle \lim_{n \to \infty} \left(1 + \frac{2}{\ln(|x|)} \frac{\ln(|x|)}{\sqrt{n}} \right) = 1, \mbox{ oraz } \lim_{n \to \infty} \sqrt{n}\ln(|x|) = -\infty $$
czyli 
$$ \lim_{n \to \infty} \frac{|x|^{\sqrt{n}}}{\frac{1}{n^2}} = 0, $$
więc dla dostatecznie dużych $n$ zachodzi $$|x|^{\sqrt{n}} < \frac{1}{n^2}$$
czyli szereg zbieżny jednostajnie z kryterium Weierstrassa na $(-1, 1)$. \newline
Oczywiście suma szeregu jest też ciągła na $(-1, 1)$, gdyż ciąg sum częściowych spełnia założenia Twierdzenia 7.7 ze skryptu Pawła Strzeleckiego. 
$$ S_N : (-1, 1) \to \mathbb{R}, \; S_N(x) = \sum_{n=1}^{N} |x|^{\sqrt{n}} $$
Każdy wyraz $S_N$ ciągły (skończona suma funkcji ciągłych). \newline
$(-1, 1)$ jest przedziałem. \newline
$S_N \rightrightarrows S \mbox{ na } (-1, 1)$ \newline
Czyli funkcja graniczna $S$ ciągła na $(-1, 1)$.


\end{document}
