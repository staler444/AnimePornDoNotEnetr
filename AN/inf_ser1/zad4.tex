\documentclass{article}
\usepackage[T1]{fontenc}
\usepackage{polski}
\usepackage[polish]{babel}
\usepackage[utf8x]{inputenc}
\usepackage{fontspec}
\usepackage{mathtools}
\usepackage{amssymb}
\usepackage[hidelinks]{hyperref}
\usepackage{amsmath,suetterl,graphicx,mathrsfs}
\usepackage[a4paper, total={7in, 10in}]{geometry}
\usepackage[skip=4pt plus1pt, indent=0pt]{parskip}

\def\vv{v_2^2}

\title{Analiza inf. II}
\author{Bartosz Kucypera, bk439964}
\date{\today}

\begin{document}

\maketitle

Rozwiązanie przygotowałem samodzielnie, ze świadomością, iż etyczne zdobywanie zaliczeń jest, zgodnie z Regulaminem Studiów, obowiązkiem studentek i studentów UW.

\section*{Zadanie 1}
Niech $f:\mathbb{R}^2 \to \mathbb{R}$ bedzie dana wzorem
$$f(x,y)=-3x^4-y^2+4yx^2.$$
Wykazać, że $h_v(t) = f(tv_1,tv_2)$ ma maksimum lokalne w $t=0$ dla każdego wektora $v$ długości $1$.

Zbadajmy pochodną $h_v$.
$$h'_v(t) = \left(-3v_1^4t^4-v_2^2t^2+4v_1^2v_2t^3 \right)\frac{d}{dt} = 
2t(-6v_1^4t^2-v_2^2+6v_1^2v_2t)$$
Korzystamy z długości, $ v_1^2 = 1-v_2^2 $.

$$h'_v(t) = 2t(-6(1-v_2^2)^2t^2-v_2^2+6(1-v_2^2)v_2t) $$
\subsection*{1) $v_2=0$}
$$h'_v(t) = 2t(-6t^2) $$
Dla $t<0$ pochodna dodatnia, zeruje się w $t=0$ i ujemna dla $t>0$, czyli $h_v$ ma w $t=0$ maksimum lokalne.
\subsection*{2) $v_2=1$}

$$h'_v(t) = -2t $$
Dla $t<0$ pochodna dodatnia, zeruje się w $t=0$ i ujemna na dla $t>0$, czyli $h_v$ ma w $t=0$ maksimum lokalne.

\subsection*{3) $0<|v_2|<1$}
W tym przypdaku pochodna jest wielomianem trzeciego stopnia z ujemnym współczynnikiem przy najwyższej potędze. Jeden z jego pierwiastków to 0. Znajdzmy pozostałe dwa. \newline 
Szukamy pierwiastów $-6(1-v_2^2)^2t^2 + 6(1-v_2^2)v_2t -v_2^2$.


$$ \Delta_t = 36(1-\vv)^2\vv -24(1-\vv)^2\vv = 12(1-\vv)^2\vv $$
$$\sqrt{\Delta_t} = 2\sqrt{3}(1-v_2^2)|v_2|$$
Niezależnie od znaku $v_2$ uzyskujemy dwa pierwiastki:
$$ x_1 = \frac{(3+\sqrt{3})v_2}{6(1-\vv)} $$
$$ x_2 = \frac{(3-\sqrt{3})v_2}{6(1-\vv)} $$
Oba są tego znaku co $v_2$.

\newpage
\subsection*{3.1) $v_2>0$}

Jeśli $v_2>0$ to $h'_v$ ma 3 pierwiastki:
$$0<x_2<x_1 $$
$\displaystyle h'_v(-1) = 2(6(1-v_2^2)^2+v_2^2+6(1-v_2^2)v_2), 0<v_2<1$, więc wszystko dodatnie, czyli $h'_v$ dodatnia dla $x<0$. \newline

\def\xd{\frac{v_2}{6(1-\vv)}}

$3-\sqrt{3} > 1$, więc $\displaystyle 0<\xd<x_2$. \newline
$$\displaystyle h'_v\left(\xd\right) = 2\xd\left(-6(1-\vv)^2\left(\xd\right)^2 -\vv + 6(1-\vv)v_2\xd  \right) = \frac{-1}{18}\frac{v_2^3}{(1-\vv)} < 0.$$ \newline
Czyli $h'_v$ ujemna na $(0,x_2)$. \newline \newline
Skoro $h'_v$ zeruje się dla $t=0$, dodatnia dla $t<0$ i ujemna na $(0,x_2)$, to $h_v$ ma w 0 maksimum lokalne.

\subsection*{3.2) $v_2<0$}
Jeśli $v_2<0$ to $h'_v$ ma 3 pierwiastki:
$$x_1<x_2<0$$
$$\displaystyle h'_v(1) = 2(-6(1-\vv)^2-\vv+6(1-\vv)v_2) < 0,$$ bo $-1<v_2<0$, więc $h'_v$ ujemne dla $x>0$. \newline \newline
Znowu, skoro $\displaystyle 3-\sqrt{3} > 1$, to $\displaystyle x_2<\xd<0$. \newline
$$h'_v\left(\xd\right) = \frac{-1}{18}\frac{v_2^3}{(1-\vv)} > 0, \mbox{ bo } v_2^3<0,$$
czyli $h'_v$ dodatnie na $(x_2,0)$.\newline \newline
Skoro $h'_v$ dodatnia na $(x_2,0)$, zeruje się dla $t=0$, oraz dodatnia dla $t>0$, to $h_v$ ma w zerze maksimum lokalne.

\subsection*{Konkluzja}
Faktycznie dla każdego wektora $v$ długości 1, $h_v$ ma w 0 maksimum lokalne. Nie możemy jednak wnioskować na tej podstawie, że $f$ ma w $(0,0)$ maksimum lokalne. \newline
Z definicji musiała by istnieć taka kula otwarta, z środkiem w $(0,0)$, że $f$ w $(0,0)$ przyjmuje większą wartość od reszty punktów w tej kuli. \newline 
Możemy jednak skonstruować ciąg o wartościach większych od $f(0, 0)$, który dąży do $(0,0)$ po spirali. Możemy tak go skonstruować by każdy z jego elementów leżał na innej prostej przechodzącej przez $(0,0)$, czyli pochodne kierunkowe dalej będą posiadać maksimum w $t=0$, każda będzie miałą maksymalnie jeden punkt tego ciągu w swojej dziedzinie a skoro ciąg dąży do $(0,0)$ to dla każdej kuli o środku w $(0,0)$, będzie istnieć nieskończenie wiele punktów dla których $f$ przyjmnie większe wartości niż w $(0,0)$.



\end{document}
