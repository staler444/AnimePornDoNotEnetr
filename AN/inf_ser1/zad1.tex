\documentclass{article}
\usepackage[T1]{fontenc}
\usepackage{polski}
\usepackage[polish]{babel}
\usepackage[utf8x]{inputenc}
\usepackage{fontspec}
\usepackage{mathtools}
\usepackage{amssymb}
\usepackage[hidelinks]{hyperref}
\usepackage{amsmath,suetterl,graphicx,mathrsfs}
\usepackage[a4paper, total={7in, 12in}]{geometry}
\usepackage[skip=4pt plus1pt, indent=0pt]{parskip}

\title{Analiza}
\author{Bartosz Kucypera, bk439964}
\date{\today}

\begin{document}

\maketitle

Rozwiązanie przygotowałem samodzielnie, ze świadomością, iż etyczne zdobywanie zaliczeń jest, zgodnie z Regulaminem Studiów, obowiązkiem studentek i studentów UW.

\section*{Zadanie 1} 
Ciąg $(a_n)_{n=1}^{\infty}$ jest dany wzorem $a_n=\sqrt[2n+1]{(2n)!}$. Niech
$$b_n=\sup_{x\in[0,a_n]} \left| \cos{x}-\sum_{k=0}^{n}\frac{(-1)^k}{(2k)!}x^{2k} \right|$$
Udownodnić, że $\displaystyle \lim_{n\to \infty}b_n=0$.

\subsection*{}
Skorzystajmy z rozwinięcia $\cos{x}$ w szereg Taylora w zerze z resztą w postaci Lagrange’a, by zapisać $b_n$ w inny, równoważny sposób.

$$ b_n = \sup_{x\in[0,a_n]} \left| \frac{\cos^{(2n+1)}(c(x))}{(2n+1)!}x^{2n+1} \right|,$$
gdzie $c(x)$ to funkcja stałej reszty Lagrange'a, $c(x)\in[0,x]$.

\subsection*{}
Oczywiście $b_n \ge 0$. Zachodzi następujący ciąg nierówności

$$ \sup_{x\in[0,a_n]} \left| \frac{\cos^{(2n+1)}(c(x))}{(2n+1)!}x^{2n+1} \right| \le 
\sup_{x\in[0,a_n]} \left| \frac{1}{(2n+1)!}x^{2n+1} \right| \le 
\frac{(a_n)^{2n+1}}{(2n+1)!} $$
czyli,

$$ 0 \le b_n \le \frac{1}{2n+1} $$
Skoro $\displaystyle \lim_{n \to \infty} \frac{1}{2n+1} = 0$, to z trzech ciągów 
$$\lim_{n \to \infty} b_n = 0$$



\end{document}
