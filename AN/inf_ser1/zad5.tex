\documentclass{article}
\usepackage[T1]{fontenc}
\usepackage{polski}
\usepackage[polish]{babel}
\usepackage[utf8x]{inputenc}
\usepackage{fontspec}
\usepackage{mathtools}
\usepackage{amssymb}
\usepackage[hidelinks]{hyperref}
\usepackage{amsmath,suetterl,graphicx,mathrsfs}
\usepackage[a4paper, total={7in, 10in}]{geometry}
\usepackage[skip=4pt plus1pt, indent=0pt]{parskip}

\title{Analiza inf. II}
\author{Bartosz Kucypera, bk439964}
\date{\today}

\begin{document}

\maketitle

Rozwiązanie przygotowałem samodzielnie, ze świadomością, iż etyczne zdobywanie zaliczeń jest, zgodnie z Regulaminem Studiów, obowiązkiem studentek i studentów UW.

\section*{Zadanie 5}

Funkcja $f$ określona na zbiorze $\{(x,y)\in \mathbb{R}^2:xy>-1\}$ wzorem

\[ 
f(x,y)=
\begin{cases} \displaystyle
	\frac{\sqrt{1+xy}-1}{y} & \mbox{ dla } y\neq0, \\
	x /2 & \mbox{ dla }y=0. \\
   \end{cases}
\]


Zbadać, czy $Df(a)$ istnieje dla $a=(0,0)$ i $a=(1,0)$.

\subsection*{1) $a=(0,0)$}
Niech $A$ będzie macieżą, pochodnych cząstkowych w punkcie $a$. \newline \newline 
$Df(a)$ istnieje wtedy i tylko wtedy gdy,  $\displaystyle \lim_{h \to (0,0)} \frac{f(a+h)-f(a)-A \cdot h}{||h||} = 0$. Jeśli istnieje, to $Df(a)$ = $A$. \newline

Obliczmy $A$. Najepierw pochodna po $x$:
$$ \lim_{h \to 0} \frac{f(h,0) - f(0, 0)}{h} = 
\lim_{h \to 0} \frac{h/2}{h} = \frac{1}{2}$$
Pochodna po $y$:
$$
\lim_{h \to 0} \frac{f(0, h) - f(0, 0)}{h} = 
\lim_{h \to 0} \frac{\frac{\sqrt{1}-1}{h}}{h} = 0
$$

Czyli
$$ A =
\begin{bmatrix}
1 /2 \\
0
\end{bmatrix}
$$
Dla każdego $v$ postaci $(v_1, 0)$, gdzie $v_1 \neq 0$, zachodzi
$$ \frac{f(a+v)-f(a)- A \cdot v}{||v||} = \frac{\frac{v_1}{2} - \frac{v_1}{2}}{||v||} = 0 $$
Teraz, podobnie dla każdego wektora $v$ postaci $(0, v_2)$ gdzie $v_2 \neq 0$, zachodzi
$$ \frac{f(a+v)-f(a)- A \cdot v}{||v||} = \frac{\frac{\sqrt{1} - 1}{v_2}}{||v||} = 0$$
Załóżmy więc, że obie współżędne wektora dążącego do $a$ są niezerowe. Wtedy
$$ \lim_{h\to (0,0)} \frac{f(a+h) - f(a)-A \cdot h}{||h||} = 
\lim_{h \to (0,0)} \frac{\frac{\sqrt{1+h_1h_2}-1}{h_2} - \frac{h_1}{2}}{||h||} = 
\lim_{h \to (0,0)}\frac{\frac{\exp(\frac{1}{2} \ln(1+h_1h_2))}{h_2} - \frac{h_1}{2}}{||h||} $$
Korzystając z rozwinięcia w szereg Taylora $\ln$ a potem $\exp$, otrzymujemy
$$ \lim_{h \to (0,0)}\frac{\frac{\exp(\frac{1}{2} \ln(1+h_1h_2))}{h_2} - \frac{h_1}{2}}{||h||} =
\lim_{h \to (0,0)}\frac{\frac{\frac{1}{2}h_1h_2 + o(h_1h_2)}{h_2} - \frac{h_1}{2}}{||h||} = 
\lim_{h \to (0,0)} \frac{o(h_1h_2)}{||h||}$$
co oczywiście zbiega do 0, bo 
$$0 \le \left| \frac{o(h_1h_2)}{||h||} \right| \le \left| \frac{o(h_1h_2)}{2h_1h_2} \right| $$
a 
$$ \lim_{h \to (0,0)} \left| \frac{o(h_1h_2)}{2h_1h_2} \right| = 0, \mbox{ z definicji.} $$

Czyli $Df(a)$ istnieje i jest równe $A$.

\subsection*{2) $a=(1,0)$}
Tak samo jak wcześniej, wyliczamy macież $A$. \newline
Pochodna cząstkowa po $x$ dalej jest stała i równa $1/2$. Policzmy pochodną po $y$ w $a$.
$$\lim_{h \to 0} \frac{f(1,0+h)-f(1,0)}{h} = 
\lim_{h \to 0} \frac{\frac{\sqrt{1+h}-1}{h} - \frac{1}{2}}{h} 
= \lim_{h \to 0}\frac{\frac{\exp(\frac{1}{2}\ln(1+h))-1}{h} - \frac{1}{2}}{h} $$
Tak jak wcześniej, korzystamy z rowinięć w szeregi Tylora, $\exp$ i $\ln$, tylko tym razem z większą dokładnością.

$$ \lim_{h \to 0}\frac{\frac{\exp(\frac{1}{2}\ln(1+h))-1}{h} - \frac{1}{2}}{h} = 
\lim_{h \to 0} -\frac{1}{8} + \frac{o(h^2)}{h^2} = -\frac{1}{8} $$
Czyli 
$$ A =
\begin{bmatrix}
\: \; \: 1 /2 \\
-1 /8
\end{bmatrix}
$$

Niech $h_n=(0,-1 /n)$, $h_n$ dąży do $(0,0)$ gdy $n$ dąży do nieskończoności. Zauważmy, że
$$ \lim_{n \to \infty} \frac{f(a+h_n) - f(a) - A \cdot h_n}{||h_n||} = 
\lim_{n \to \infty} \frac{-\frac{1}{2} - \frac{1}{8n}}{\frac{1}{n^2}} =
\lim_{n \to \infty} -\frac{1}{2}n^2 - \frac{1}{8}n =
-\infty
$$
czyli $Df(a)$ nie istnieje.


\end{document}
