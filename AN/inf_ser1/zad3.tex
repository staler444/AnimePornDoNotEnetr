\documentclass{article}
\usepackage[T1]{fontenc}
\usepackage{polski}
\usepackage[polish]{babel}
\usepackage[utf8x]{inputenc}
\usepackage{fontspec}
\usepackage{mathtools}
\usepackage{amssymb}
\usepackage[hidelinks]{hyperref}
\usepackage{amsmath,suetterl,graphicx,mathrsfs}
\usepackage[a4paper, total={7in, 10in}]{geometry}
\usepackage[skip=3pt plus1pt, indent=0pt]{parskip}

\title{Analiza inf. II}
\author{Bartosz Kucypera, bk439964}
\date{\today}

\begin{document}

\maketitle

Rozwiązanie przygotowałem samodzielnie, ze świadomością, iż etyczne zdobywanie zaliczeń jest, zgodnie z Regulaminem Studiów, obowiązkiem studentek i studentów UW.

\section*{Zadanie 3} 
$$ F(x) = \int_x^{3x} \frac{\ln(t+1)}{t}dt, \mbox{ dla } x>0 $$

\subsection*{1) F rosnąca:}
Zauważmy, że:

$$ F(x) = G(3x) - G(x) $$
dla pewnej funckji $G$ takiej, że $G'(x)=\frac{\ln(1+x)}{x}$. \newline 

Stąd: 
$$ F'(x) = G'(3x) - G'(x) = \frac{\ln(1+3x)}{x} - \frac{\ln(1+x)}{x} $$
ln ściśle rosnący oraz $x>0$, więc 
$$ \ln(1+3x) > \ln(1+x) $$
czyli $F' > 0$, czyli $F$ rosnąca na $(0,\infty)$.

\subsection*{2) $\displaystyle \lim_{x \to \infty} F(x)$}
Zauważmy, że 

$$ \left( \frac{\ln(1+x)}{x} \right)' = \frac{\frac{x}{1+x} - \ln(1+x)}{x^2}$$

$\frac{x}{1+x} \le \ln(1+x)$ dla $x>-1$, czyli

$$ \left( \frac{\ln(1+x)}{x} \right)' \le 0 \mbox{ dla } x>0$$
czyli $\frac{\ln(1+x)}{x}$ nierosnąca na $(0,\infty)$.

Korzystając z geometrycznej interpretacji całki, możemy teraz stwierdzić, że:

$$ F(x) = \int_x^{3x} \frac{\ln(1+t)}{t}dt \ge 2x \cdot \frac{\ln(1+3x)}{3x} = \frac{2}{3}\ln(1+3x) $$

$\displaystyle \lim_{x \to \infty} \frac{2}{3} \ln(1+3x) = \infty$, więc też 

$$ \lim_{x \to \infty} F(x) = \infty $$

\newpage

\subsection*{3) $\displaystyle \lim_{x \to 0} F(x)$}
Kożystając z gemetrycznej interpretacji całki, oraz z tego, że funkcja podcałkowa nierosnąca otrzymujemy następującą nierówność:

$$ \int_x^{3x} \frac{\ln(1+t)}{t}dt \le 2x \cdot \frac{\ln(1+x)}{x} = 2\ln(1+x) $$
Ponadto $F(x) \ge 0$ ponieważ, funckja podcałkowa dodatnia oraz $3x > x$ na $(0, \infty)$. Skoro 

$$ 0 \le F(x) \le 2\ln(1+x) \mbox{ oraz, } \lim_{x \to 0} 2\ln(1+x) = 0$$
to
$$ \lim_{x \to 0} F(x) = 0 $$

\end{document}
