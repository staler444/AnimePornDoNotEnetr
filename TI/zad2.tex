\documentclass{article}
\usepackage[T1]{fontenc}
\usepackage{polski}
\usepackage[polish]{babel}
\usepackage[utf8x]{inputenc}
\usepackage{fontspec}
\usepackage{mathtools}
\usepackage{amssymb}
\usepackage[hidelinks]{hyperref}
\usepackage{amsmath,suetterl,graphicx,mathrsfs}
\usepackage[a4paper, total={6.2in, 10in}]{geometry}
\usepackage[skip=4pt plus1pt, indent=0pt]{parskip}

\title{Kolokwium z Teori Informacji}
\author{Bartosz Kucypera}
\date{\today}

\begin{document}

\maketitle

\section*{Zadanie 2}

\subsection*{a)}
Dowieść, że dla zmiennych losowych $X,Y,Z$ i funckji $f,g$, zachodzi

$$I(f(X);g(Y)|Z) \le I(X;Y|Z)$$

Zauważmy, że możemy stworzyć pomocniczą zmienną losową $A = Y|Z$, i zapisać

$$I(X;Y|Z) = I(X;A)$$

Wtedy z "twierdzenia o przetwarzaniu informacji" mamy:

$$I(X;A) \ge I(f(X);A) = I(f(X);Y|Z)$$

teraz, niech $B = f(X)|Z$

$$ I(f(X);Y|Z) = I(Y;f(X)|Z) = I(Y;B) \ge I(g(Y);B) = I(g(Y);f(X))$$

czyli zachodzi

$$ I(f(X);g(Y)|Z) \le I(X;Y|Z) $$

\subsection*{b)}
Podać przykład, kiedy 

$$ I(f(X);g(Y)|Z) < I(f(X);Y|Z) < I(X;Y|Z) $$

Niech $X$ zwraca ciąg 42 bitów z rozkładem jednostajnym.

Niech $Y$ zwraca ciąg 42 bitów z rozkładem jednostajnym.

Niech $Z= X\oplus Y$ (xor).

Niech $f((b_1, b_2, \cdots, b_n)) = (0, b_2, b_3, \cdots b_n)$ (zeruje pierwszy bit).

Niech $g((b_1, \cdots, b_n)) = (0, \cdots, 0)$ (zeruje wszystkie bity).

\subsection*{Pierwsza nierówność}

$$I(f(X);Y|Z) < I(X;Y|Z)$$

$$H(f(X)|Z) - H(f(X)|Y,Z) < H(X|Z) - H(X|Y,Z)$$

Zauważmy, że $H(X|Y,Z) = 0$ oraz $H(f(X)|Y,Z) = 0$ (znając $Y$ i $Z$ znamy $X$ oraz $f(X)$).

$$H(f(X)|Z) < H(X|Z)$$

Teraz zauważmy, że znajomośc $Z$ w żadne sposób nie ogranicza $X$ czyli $H(X|Z) = H(X)$,
$H(f(X)|Z) = H(f(X))$.

$$H(f(X)) < H(X)$$
co jest oczywiście prawdą ($f(X)$ traci możliwość wyboru pierwszego bitu).

Czyli faktycznie pierwsza nierówność zachodzi.

\subsection*{Druga nierówność}

$$I(f(X);g(Y)|Z) < I(f(X);Y|Z)$$

Mamy 

$$I(f(X);g(Y)|Z) = H(f(X)|Z) - H(f(X)|Z,g(Y))$$

Skoro $g$ zawsze zwraca ciąg samych zer to $H(f(X)|Z,g(Y)) = H(f(X)|Z)$

Czyli 
$$I(f(X);g(Y)|Z) = 0$$

Natomiast prawa strona:

$$I(f(X);Y|Z) = H(f(X)|Z) - H(f(X)|Y,Z) = H(f(X)|Z) = H(f(X)) > 0$$

czyli nierównośc zachodzi.

\end{document}:
