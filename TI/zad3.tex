\documentclass{article}
\usepackage[T1]{fontenc}
\usepackage{polski}
\usepackage[polish]{babel}
\usepackage[utf8x]{inputenc}
\usepackage{fontspec}
\usepackage{mathtools}
\usepackage{amssymb}
\usepackage[hidelinks]{hyperref}
\usepackage{amsmath,suetterl,graphicx,mathrsfs}
\usepackage[a4paper, total={6.2in, 10in}]{geometry}
\usepackage[skip=4pt plus1pt, indent=0pt]{parskip}

\title{Kolokwium z Teori Informacji}
\author{Bartosz Kucypera}
\date{\today}

\begin{document}

\maketitle

\section*{Zadanie 3}

Który z kanałów ma większą przepustowość? 

$$ \begin{pmatrix}
0 & \frac{1}{3} & \frac{1}{3} & \frac{1}{3} \\
\frac{1}{3} & 0 & \frac{1}{3} & \frac{1}{3} \\
\frac{1}{3} & \frac{1}{3} & 0 & \frac{1}{3} \\
\frac{1}{3} & \frac{1}{3} & \frac{1}{3} & 0 
\end{pmatrix}  $$

czy 

$$ \begin{pmatrix}
0 & \frac{1}{3} & \frac{1}{3} & \frac{1}{3} \\
0 & 0 & \frac{1}{3} & \frac{2}{3} \\
\frac{1}{3} & \frac{2}{3} & 0 & 0 \\
\frac{1}{3} & \frac{1}{3} & \frac{1}{3} & 0 
\end{pmatrix}  $$

Policzmy przepustowość pierwszego kanału:

Niech $X$ wejściem kanału, a $Y$ wyjściem.

Chcemy znaleźć takie $X$, że $I(X;Y)$, maksymalne.

$$I(X;Y) = H(Y) - H(Y|X)$$

Jeśli znamy wejście to, łatwo policzyć entropie wyjscia, $H(Y|X) = \log_2(3)$.

Teraz, ponieważ wiemy, że $\log(n)$ to górne ograniczenie entropi (dla zmiennej losowej na zbioże n elementowym) osiągane gdy zmienna losowa ma rozkład jednostajny, możemy tak dobrać $X$ by $H(Y)$ było maksymalne.

Biorąc $X$ o rozkładzie jednostajnym uzyskujemy jednostajny rozkład $Y$.

$$p(y_i) = \sum_{x_j} p(x_j) \cdot p(x_j, y_i) = 3 \cdot \frac{1}{3} \cdot \frac{1}{4} + 9\cdot \frac{1}{4} = \frac{1}{4}$$

Czyli dla $X$ o rozkładzie jednostajnym pierwszy kanał osiąga maksymalną przepustowość, $\log_2(4)-\log_2(3) = 2 - \log_2(3)$.

Sprawdźmy więc jaką przepustowość osiągnie drugi kanał na tym samym rozkładzie.

Niech $Y'$ wyjściem drugiego kanału.

$$I(X;Y') = H(X) - H(X|Y')$$

$H(X) = 2$, $H(X|Y')$ rozpiszę z entropi warunkowej:

$$ H(X|Y') = \sum_{y\in Y'} p(y) \cdot H(X|Y'=y)$$

$$H(X|Y') = \frac{1}{4} \cdot \left( \log_2(2) + \frac{1}{2}(\log_2(2) + \log_2(4)) + \log_2(3) + \frac{1}{3}\log_2(3) + \frac{2}{3}\log_2(3/2) \right) = \frac{1}{2}\log_2(3)+\frac{11}{24}$$

Sprawdzamy dla którego kanału wspólna informacja wejścia i wyjścia jest większa:

$$2 - \log_2(3) \;\; ? \;\;  2 - \frac{1}{2}\log_2(3) - \frac{11}{24}$$

$$ \frac{11}{24} \;\; ? \;\; \frac{1}{2}\log_2(3) $$

$$ \frac{11}{12} \;\; ? \;\; \log_2(3) $$

Tu już wiemy że lewa strona jest mniejsza (bo jest mniejsza od 1 a prawa większa od 1).

czyli zachodzi

$$I(X, Y) < I(X, Y')$$

a ponieważ dla $X$ pierwszy kanał osiągał maksymalną przepustowość, to na pewno drugi kanał ma większą przepustowość od pierwszego (potencjalnie jeszcze lepszą niż dla tego rozkładu).








\end{document}
