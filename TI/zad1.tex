\documentclass{article}
\usepackage[T1]{fontenc}
\usepackage{polski}
\usepackage[polish]{babel}
\usepackage[utf8x]{inputenc}
\usepackage{fontspec}
\usepackage{mathtools}
\usepackage{amssymb}
\usepackage[hidelinks]{hyperref}
\usepackage{amsmath,suetterl,graphicx,mathrsfs}
\usepackage[a4paper, total={6.2in, 10in}]{geometry}
\usepackage[skip=4pt plus1pt, indent=0pt]{parskip}

\title{Kolokwium Teoria Informacji}
\author{Bartosz Kucypera}
\date{\today}

\begin{document}

\maketitle

\section*{Zadanie 1}
Niech $n-$ ilość monet, $k-$ wybrany przez nas podział ważenia. \newline
(dodatkowo niech $n,k$ "sęsowne" czyli $n \ge 2, 1\le 2k\le n$,  $\:n,k \in \mathbb{N}$)

Proces z zadania możemy zasymulować poprzez dwie zmienne losowe.

X - zmienna losowa wybierająca lżejszą monetę.

Y - zmienna symulująca ważenie, zmienia rozmiar zbioru monet które mogą okazać się najlżejszą.

X ma rozkład jednostajny (każda moneta z takim samym prawdopodobieństwem może okazać się najlżejsza).

Y z prawdopodobieństwem $\frac{2k}{n}$ zmniejszy zbiór potencjalnie najlżejszych monet do k (jeśli ważone zbiory nie będą takie same, to wiemy, że w lżejszym z nich znajduje się lżejsza moneta) i z prawdopodobieństwem $\frac{n - 2k}{n}$ zmniejszy ten zbiór do $n-2k$ (jeśli ważone zbiory są takie same to lżejsza moneta jest w zbioże monet nieważonych).

Chcemy zmaksymalizować wspólną inforamcję $X, Y$, czyli informację jaką uzyskamy o najlżejszej monecie po ważeniu.

Mamy $I(X;Y)=H(X) - H(X|Y)$, przy czym $H(X)$ jest stałe i równe $\log_2(n)$, więc zadanie sprowadza się do zminimalizowania $H(X|Y)$.

Rozpiszmy ze wzoru na entropię wrunkową zmiennej dyskretnej:

$$H(X|Y) = \sum_{y \in Y}p(y)H(X|Y=y) = \frac{2k}{n} \log_2(k) + \frac{n- 2k}{n} \log_2(n-2k)$$
Zkorzystaliśmy z wcześniej rozpisanego rozkładu $Y$ i z tego, że jeśli rozmiar zbioru monet potencjalnie najlżejszych to $r$, to ponieważ $X$ ma rozkład jednostajny na tym zbioże, będzie miało entropię równą $\log_2(r)$.

W naszym zadaniu $n$ jest stałą, zastanawiamy się nad wyborem optymalnego k.

Niech, więc $f: (0, n/2) \to \mathbb{R}$ dana wzorem $f(x) = \frac{2x}{n}\log_2(x)+\frac{n-2x}{n}\log_2(n-2x)$.

Znalezienie minimum funckji $f$, (jeśli ono istnieje) rozwiązuje zadanie, (z dokładnością do tego, że k musi być liczbą naturalną, zajmiemy się tym potem).

Pochodna $f$ po $x$:

$$ f'(x) = \frac{2}{n}\log_2(x) - \frac{2}{n}\log_2(n-2x) $$

Szukamy miejsc zerowych:

$$f'(x) = 0$$
$$\frac{2}{n}\log_2(x) - \frac{2}{n}\log_2(n-2x) = 0$$
$$\log_2(x) = \log_2(n-2x)$$
$$x = n-2x$$
$$x = \frac{n}{3}$$

Przy czym
$$f'(1) = -\frac{2}{n}\log_2(n-2) \le 0$$
$$f'((n-1)/2) = \frac{2}{n}\log_2((n-1)/2) > 0$$
czyli w punkcie $x = \frac{n}{3}$ $f$ osiaga minimum.

Musimy jeszcze zagwarantować by $k$ było liczbą całkowitą.

Jeśli $n$ przystaje do 0 modulo 3, to optymalne $k=\frac{n}{3}$ i już.

Jeśli jest inaczej to mamy dwóch kandydatów na $k$, sufit/podłogę z $\frac{n}{3}$.

Sprawdźmy, więc w każdym przypadku który z tych kandydatów jest lepszy.

\subsection*{Przypadek $n = 3c + 1$}
Mamy dwóch kandydatów:
$k_1$ = $c$ i $k_2$ = $c+1$.

Prównajmy, który da nam mniejszą entropię warunkową:

$$ f(k_1) \:?\: f(k_2) $$

$$  \frac{2c}{3c+1}\log_2(c) + \frac{3c+1-2c}{3c+1}\log_2(3c+1-2c) \: ? \: 
    \frac{2c + 2}{3c+1}\log_2(c+1) + \frac{3c+1-2c-2}{3c+1}\log_2(3c+1-2c-2) $$

$$ 2c\log_2(c) + (c+1)\log_2(c+1) \: ? \: 2(c+1)\log_2(c+1) + (c-1)\log_2(c-1) $$

$$2c\log_2(c) \: ? \:(c+1)\log_2(c+1)+(c-1)\log_2(c-1) $$

$$ c\log_2(c) - (c-1)\log_2(c-1) \: ? \: (c+1)\log_2(c+1) - c\log_2(c)$$

teraz, wiemu już, że nierówność jest następująca:

$$ c\log_2(c) - (c-1)\log_2(c-1) \: < \: (c+1)\log_2(c+1) - c\log_2(c)$$

wynika to z tego, że pochodna funckji $x\log_2(x)$ jest ściśle rosnąca, ($(x\log_2(x))' = \frac{\ln(x)+1}{\ln(2)}$).

Czyli zachodzi:
$$ f(k_1) < f(k_2)$$

Czyli optymalne $k$ to $\lfloor \frac{n}{3} \rfloor$.

\newpage 

\subsection*{Przypadek $n = 3c+2$}
Mamy dwóch kandydatów:
$k_1 = c$ i $k_2 = c+1$.

Prównajmy który da nam mnijeszą entropię warunkową:

$$f(k_1) \: ? \: f(k_2) $$

$$  \frac{2c}{3c+2}\log_2(c) + \frac{3c+2-2c}{3c+2}\log_2(3c+2-2c) \: ? \: 
    \frac{2c + 2}{3c+2}\log_2(c+1) + \frac{3c+2-2c-2}{3c+2}\log_2(3c+2-2c-2) $$

$$ 2c\log_2(c) + (c+2)\log_2(c+2) \: ? \: 2(c+1)\log_2(c+1) + c\log_2(c) $$

$$c\log_2(c) + (c+2)\log_2(c+2) \: ? \: 2(c+1)\log_2(c+1) $$

$$ (c+2)\log_2(c+2) - (c+1)\log_2(c+1) \: ? \: (c+1)\log_2(c+1) - c\log_2(c) $$

tak jak wcześniej, wiemy już, że zachodzi:


$$ (c+2)\log_2(c+2) - (c+1)\log_2(c+1) \: > \: (c+1)\log_2(c+1) - c\log_2(c) $$

czyli

$$ f(k_1) > f(k_2) $$

Czyli optymalne $k$ to $ \lceil \frac{n}{3} \rceil$.

\subsection*{Przypadek szczególny zadania}

W podstawowej treści mieliśmy $n = 7$, czyli $n = 3 \cdot 2 + 1$.

Czyli wiemy, że optymalne $k$ to $\lfloor \frac{n}{3} \rfloor = 2$.

Prawidłowa jest, więc odpowiedź 2., ważymy dwie przeciwko dwóm.



\end{document}:
